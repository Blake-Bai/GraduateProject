\chapter{实验与结论} \label{chap:exandcon}
我们在各种不同的2维区域上进行了实验,验证了算法的有效性,并和(ref ma-quad,ref  paving,ref sdu-automesh)进行了比较,最后提出了把算法扩展到3维情形生成六面体网格的展望。
\section{实验与比较}
我们的实验在一台3.4GHz CPU, 8GB RAM的电脑上运行。\ref{sec:control-singular}节展示了不同参数下算法产生的多正N边形结构以及四边形网格不同的奇异结构。\ref{sec:comparison}节比较了本文的方法与(ref)的方法产生的四边形网格。
\subsection{可控的奇异性} \label{sec:control-singular}
输入不同的N以及高斯核宽度\mksize,我们的算法将2维网格变形为不同的多正N边形结构,并由此产生奇异性可控的四边形网格。由于奇异点的数目和四边形网格的质量往往是相互矛盾的,即:为了更好的逼近连续的边界,较高质量的四边形网格往往需要较多的奇异点,所以算法对奇异性的控制非常重要。

\todo{图}展示了不同的N对tool模型变形得到的多正N边形结构以及用中轴线分割方法产生的四边形网格,高斯核宽度$\mksize$均为4。tool模型的法向几乎是均匀分布的,故N越大,变形时的目标法向越多,对原始模型的逼近越好,产生的奇异点也越多。
\begin{figure}[t]
	\centerline
	{
		\begin{overpic}[width=0.99\columnwidth]{N246_tool}
			\put(13,-3){\textbf{tool}}
			\put(33,-3){\textbf{N=2}}
			\put(58,-3){\textbf{N=4}}
			\put(88,-3){\textbf{N=6}}
		\end{overpic}
	}
	\vspace{4mm}
	\caption{tool模型产生的不同的多正N边形结构以及对应的四边形网格}
	\label{fig:N246}
	\vspace{-3mm}
\end{figure}

\todo{图}展示了不同的高斯核宽度$\mksize$产生的结果,N取8。$\mksize$越大,变形过程中对网格边界的光滑程度越大,且网格标记时会吞掉越长的边,因此得到的多正八边形结构对原始模型的简化程度越大,奇异点也越少。 

\begin{figure}[t]
	\centerline
	{
		\begin{overpic}[width=0.99\columnwidth]{vark}
			\put(13,-3){\textbf{guitar}}
			\put(33,-3){\textbf{$\mksize=1$}}
			\put(58,-3){\textbf{$\mksize=3$}}
			\put(88,-3){\textbf{$\mksize=5$}}
		\end{overpic}
	}
	\vspace{4mm}
	\caption{不同的$\mksize$产生的guitar模型的多正八边形结构及其四边形网格}
	\label{fig:N246}
	\vspace{-3mm}
\end{figure}

\subsection{高质量且奇异点较少的四边形网格}\label{sec:comparison}
我们把多正N边形辅助生成的四边形网格模型分别和直接采用中轴线分割的方法(ref)、前沿推进法(ref)以及法(ref)的结果做了比较。\todo{表}统计了四边形网格所有单元的缩放Jacobian最大值($J_{max}$)、最小值($J_{min}$)、平均值($J_{avg}$),角度的标准差$\sigma_{angle}$以及奇异点的数量($\#Singularity$)。\todo{图}展示了的结果,颜色越浅表示缩放Jacobian值越接近理想值1。

通过比较可以看出,我们的结果均能有效地产生较少的奇异点,同时产生质量
\begin{table}[t]
	\caption{几种方法生成四边形网格质量的比较}
		\centering \scalebox{0.8}
		{
			\begin{tabular}{lrrll}
				\toprule
				模型  & \#顶点 / \#四边形 & $J_{min}/J_{avg}$ & $\sigma_{angle}$ & \#singularity \\
				\midrule
				cactus-PQ & 432/347 &	0.56/0.94 &	14.36 &	37\\
				cactus-LQ & 402/323 & 	0.58/0.94 &	13.83 & 40\\
				cactus-MA & 617/544	&   0.17/0.88 &	23.48&	9\\
				cactus-Ours& 613/520&	0.40/0.96 &	11.26&	4\\
				\midrule
				dino-PQ &931/812&	0.38/0.95&	 13.56&	85\\
				dino-LQ &1151/1024	&0.38/0.95	&12.8&	51\\
				dino-MA&1377/1248&	0.31/0.92&	18.33	&16\\
				dino-Ours&923/825&	0.32/0.96&	12.16&	11 \\
				\midrule
				lake-PQ&569/480&	0.31/0.89&	19.23&	84\\
				lake-LQ& 1009/891	& 0.51/0.93& 14.73&159\\
				lake-MA&995/877	&0.43/0.93&	16.35&	16\\
				lake-Ours&655/562&	0.31/0.94&	14.6&	7\\
				
				\midrule
				shark&&&&\\
				\midrule
				woody&&&&\\
				\midrule
				guitar-PQ&926/797&	0.34/0.95&	13.18&	84\\
				
				guitar-LQ&550/458&	0.48/0.93&14.87	&82\\
				
				guitar-MA&1053/920&	0.15/0.95&	13.09&	10\\
				
				guitar-Ours&845/756&	0.28/0.94&	16.17&	4\\
				
				
				\bottomrule
			\end{tabular} 
		}
		\label{table:comparedata}
		\vspace{-1mm}
		
\end{table}	
\begin{figure}[t]
	\centerline
	{
		\begin{overpic}[width=0.99\columnwidth]{lake}
			\put(13,-3){\textbf{paving}}
			\put(33,-3){\textbf{looping}}
			\put(58,-3){\textbf{medial axis}}
			\put(88,-3){\textbf{Ours}}
		\end{overpic}
	}
	\vspace{4mm}
	\caption{各算法对lake模型得到的四边形网格}
	\label{fig:lark}
	\vspace{-3mm}
\end{figure}
	
\section{总结与展望}