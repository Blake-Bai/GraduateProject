\chapter{快速鲁棒的多立方体结构生成} \label{chap:polycube}
输入一个四面体网格$\mM$,本章目标是从$\mM$出发,构造一个奇异性可控的多立方体结构和一个低形变的体映射。对于曲面输入,本章使用TetGen软件~\cite{Si2015}将它们四面体化。在~\ref{sec:deform}节中我们优化表面三角形法向的光滑和对齐能量来驱动网格变形和自动去除极限点。在~\ref{sec:label_and_flatten}节中我们给出如何对网格的表面三角形进行标记,如何确定当前的网格是不是拥有一个正确的多立方体拓扑结构,以及如何使用保证无翻转的算法将网格严格压平,而得到最终的多立方体结构。

\section{网格变形} \label{sec:deform}
我们设计了三个能量函数项来驱动网格变形(~\ref{sec:obj}节):法向光滑能量,法向对齐能量和刚性形变函数~\cite{Fu2015}。在最小化的过程中,我们动态地调节上述能量函数中的比例参数,将输入网格变形成多立方体结构(~\ref{subsec:smooth} 和~\ref{subsec:align}节)。在\ref{subsec:minimize}节中,描述了如何高效地优化上述的能量函数。

\subsection{目标函数设计} \label{sec:obj}
设输入网格为$\mM$,包含$N$个四面体$\{\mt_1,\ldots,\mt_N\}$,$N_v$个顶点$\{\mv^0_1,\ldots,\mv^0_{N_v}\}$,它的边界曲面网格是$\mathcal{S}$。$\mathcal{S}$包含$n$个面$\{\mf_1,\ldots,\mf_n\}$,它们的法向是$\{\mn_1,\ldots,\mn_n\}$。我们引入一个算子$\Ax(\cdot)$,它将一个三维向量映射到离它最近的轴方向上($(\pm 1,0,0)^T$, $(0,\pm 1,0)^T$ 或者 $(0,0,\pm 1)^T$), 比如,$\Ax ((3,2,1)^T) = (1,0,0)^T$。如果所有的面法向和轴方法能够完美的对齐,那么$\mn_i = \Ax (\mn_i), \forall \, i$。已经有方法利用法向与最近的轴$\Ax(\mn_i)$之间的距离来定义的能量,驱动网格的变形,同时将网格分割成块~\cite{Gregson2011}来生成多立方体结构。因为$\Ax(\mn_i)$操作算子具有过强的局部性质,直接使用可能会导致相邻法向之间的不一致性。为了避免直接使用$\Ax(\mn_i)$,本章使用高斯光滑后的法向来驱动变形。我们定义如下的\emph{法向光滑能量}。
\begin{equation} \label{eqn:smooth}
E_s = \sum_{i=1}^n \mu_i \cdot \| \mn_i - \Ax(G_\sigma(\mn_i)) \|^2.
\end{equation}
其中$\mu_i$是$\mf_i$上的权,$G_\sigma (\cdot)$是一个高斯函数:
\begin{equation}
G_\sigma (\mn_i) = \sum_{\mf_j \in \mF} \area(\mf_j)\,\exp\left(-\dfrac{\|\mp_{\mf_i}-\mp_{\mf_j}\|}{2\sigma_s^2} \right) \cdot \mn_j.
\end{equation}
其中$\mp_{\mf_i}$是面$\mf_j$的面心,$\|\mp_{\mf_i}-\mp_{\mf_j}\|$是从$\mp_{\mf_i}$到$\mp_{\mf_j}$的测地距离的简单近似,$\sigma_s$是高斯函数的核宽度,对于所有的$\mf_i$都是一样的。通过调节高斯函数的核大小,$E_s$能够改变最终多立方体结构的奇异性。

设$n^x, n^y, n^z$是一个法向的三个分量。 因为和轴对齐的法向有且仅有两个0分量,我们提出了如下的\emph{法向对齐能量}:
\begin{equation}
E_a = \sum_{i=1}^n \nu_i \cdot \left( (n_i^x \cdot n_i^y)^2 + (n_i^y \cdot n_i^z)^2 + (n_i^z \cdot n_i^x)^2 \right).
\end{equation}
其中$\nu_i$是$\mf_i$上的权。

由于我们希望输入网格和多立方体网格之间是尽可能一样的,于是需要考虑刚性形变。为了获得低形变的映射,我们使用了AMIPS能量~\cite{Fu2015}(第\ref{chap:AMIPS}章),它能惩罚最大的形变和防止翻转的或者退化的四面体。AMIPS通过如下的方式定义映射的刚性形变。设$\mA$ 是从输入四面体$\triangle \mv^0_p\mv^0_q\mv^0_r\mv^0_s$ 到变形四面体$\triangle \mv_p\mv_q\mv_r\mv_s$的仿射变换:
\begin{equation}
\mA = \left[\mv_p-\mv_q \,|\, \mv_p-\mv_r\,|\,\mv_p-\mv_s\right] \cdot \left[\mv^0_p-\mv^0_q \,|\, \mv^0_p-\mv^0_r\,|\,\mv^0_p-\mv^0_s\right]^{-1}.
\end{equation}
共形形变定义为:
\begin{equation}
\delta_{conf} = \frac{1}{8}\left(\|\mA\|^2_F \cdot \|\mA^{-1}\|^2_F -1\right),
\end{equation}
和体积形变定义为:
\begin{equation}
 \delta_{vol} = \frac{1}{2} \left(\det \mA + (\det \mA)^{-1} \right).
\end{equation}
这里的$\|\cdot\|_F$是Frobenius范数。将上述两个形变合并得到刚性形变,\emph{刚性AMIPS能量}定义为:
\begin{equation}
E_{iso} = \sum_{i=1}^N \exp \left( s \cdot \left(\alpha \, \delta_{conf} + (1-\alpha) \, \delta_{vol} \right) \right).
\end{equation}
在我们的方法中,我们设$s=1$和$\alpha = 0.5$。 很显然,当任何的四面体退化时,AMIPS能量会趋向于无穷。因为使用了指数函数,AMIPS能够非常有效的惩罚最大的形变和产生均匀的分布形变。

\textbf{目标函数}\, 我们将上述的能量函数合并得到多立方体变形能量:
\begin{equation}
E := E_{iso} + E_s + E_a.
\end{equation}
%其中$\mu_i,\nu_i$是$E_s$和$E_a$的权。%注意到模型的朝向会影响$E$的值,因此我们需要在顶点位置上引入了一个全局旋转矩阵$R$来减少目标函数。

如果我们能够最小化$E$(变量是网格$\mM$的顶点位置),并且使$E_a$达到0,就能自动地得到多立方体结构。但是由于$E$的高度非线性,直接优化$E$不能保证$E_a=0$满足。~\cite{Gregson2011,Livesu2013}方法中的思想是,首先将曲面网格分割成和轴对齐的块,然后将这些和轴对齐后的块严格压平。本文也是利用了这个思想。也就是说在网格变形的过程中,我们不需要做到完美的$E_a=0$,只需要能给每个$\mf_i$确定一个的标记(标记共有六种可能,即六个轴中的一个),使得整个网格表面的标记集是正确的(\emph{正确标记集}的定义见\ref{subsec:label}节)。在意识到上述优化的困难和整个表面只需要正确的标记集后,我们提出了如下的优化框架:
\begin{enumerate}
    \item \emph{法向光滑变形},设$\nu_i=0$,优化$E_{iso} + E_s$使得表面的法向和轴尽量对齐,这个过程和~\cite{Gregson2011}中旋转驱动的网格变形类似,所以会产生极限点;
    \item \emph{法向对齐变形},为了处理上述过程中出现的极限点,我们设$\mu_i=0$,优化$E_{iso} + E_a$使得出现在极限点附近的区域自动和轴对齐,因而可以自动的消除极限点;
    \item  检查上述优化是不是已经产生了正确的标记集,如果是就进行网格压平,否则回到步骤1进行迭代,直至产生正确的标记集。
\end{enumerate}
下面的章节将具体介绍上述的优化过程和参数($\mu_i,\nu_i$)的调节方法。

\subsection{法向光滑变形} \label{subsec:smooth}
这一步网格变形的过程中,我们希望映射是低形变的,同时让网格表面法向和轴尽可能的对齐。为了达到这个目标,需要一个自适应的$\mu$来维持$E_{iso}$和$E_s$之间的平衡,不让$E_{iso}$或者$E_s$来主导整个能量函数。当某个能量主导整个能量函数时,要么会导致法向和轴对齐不好,要么会产生很大的形变,这都会对后续的优化产生不好的影响。对于任意一个表面三角形$\mf_i$,有且仅有一个与之相邻的四面体$\mt_i$存在,这样的四面体称为边界四面体。$\mu_i$对于每个$\mf_i$ 是不同的。在网格变形前,$\mu_i$ 被定义为:
\begin{align}
%\begin{split}
\alpha_i &= \frac{E_{iso, i}}{E_{s, i}}, \\
\mu_i &= \min \left( \max \left( \alpha_i, \mu_{\min} \right) , \mu_{\max} \right),
%\end{split}
\end{align}
其中$E_{iso, i}$是变形前边界四面体$\mt_i$上的形变,$E_{s, i}$是变形前表面三角形$\mf_i$上的法向光滑能量。如果一开始形变比较小,这时计算得到$\alpha_i$比较小,优化的过程如果出现形变能量变大,法向光滑能量降低,会导致在后续优化中$E_{iso,i}$主导整个局部能量,使得表面三角形不能和轴对齐,因此需要最小值$\mu_{\min}$来截断,避免这种情况的出现。如果一开始的形变比较大,这时计算得到的$\alpha_i$ 比较大,因此后续优化中可能会出现数值不稳定,导致优化快速的陷入局部极小,所以使用$\mu_{\max}$来避免$\mu_i$太大。在我们的实现中,设$\mu_{\min} = 0.1 \cdot \lambda$和$\mu_{\max} = 2 \cdot \lambda$,其中$\lambda$是根据$\alpha_i$的平均值确定的。因为$E_{s, i}$的最小值是0,而$E_{iso, i}$的最小值是$e^s$,所以在计算$\alpha_i$时,太小的$E_{s, i}$会导致数值不稳定和太大的$\alpha_i$平均值;太大的$E_{s, i}$表明一开始法向和轴方向的差距很大,而最终目标是要和轴方法对齐的,假如这样的三角形用于估计参数,就会使计算出的$\alpha_i$ 的平均值偏小。所以我们需要使用$ \bar{\alpha} = \frac{1}{n} \sum_{i=1}^n \alpha_i$ 对$\alpha_i$进行过滤,选择$\alpha_i \in \left[ 2 \cdot \bar{\alpha}, 0.1 \cdot \bar{\alpha} \right]$。 使用过滤后的$\alpha_j$ 的平均值来确定$\lambda$:
\begin{equation}
\lambda = \min \left( \frac{1}{m} \sum_{j=1}^m \alpha_j , \lambda_{\max} \right),
\end{equation}
其中$\lambda_{\max} = 10^{16}$用来避免数值问题,$m$是过滤后$\alpha_j$的数量。

\subsection{法向对齐变形} \label{subsec:align}
为了优化$E_{iso} + E_a$,我们首先要消除模型朝向带来的影响,选取好的朝向让尽量多的表面三角形法向和轴对齐,因此在顶点位置上引入了一个全局旋转矩阵$R$来降低目标能量函数的值。目标函数中只有$E_a$是和$R$相关的,设所有的$\nu_i=1$得到$E_a$。选择欧拉角表示$R$,优化的时候只有三个变量(三个欧拉角),因此我们选用LBFGS 算法进行优化,速度非常快。将$R$应用到顶点位置上,就可以尽量降低模型朝向带来的影响。

\textbf{讨论1}\, 上述过程和~\cite{Huang2014}中的消除朝向带来的影响的做法是类似的。但是在我们的实验中,选择欧拉角表达$R$,而~\cite{Huang2014}使用九个数表示矩阵$R$,在优化中使用软约束来使$R$接近一个旋转矩阵。我们的做法更加直接、变量更少,能保证优化出来的$R$一定是旋转矩阵。

在消除了模型朝向带来的影响后,我们需要去优化$E_{iso} + E_a$,从而消除基于法向光滑能量的网格变形带来的极限点。在优化前,需要确定$\nu_i$。$\nu_i$的更新方式和$\mu_i$是类似,只是把$E_{s, i}$换成$E_{a,i}$。优化结束后,我们使用\ref{subsec:label} 节中的标记方法对每个$\mf_i$进行标记,如果整个网格拥有标记集,跳出网格变形步骤。

\subsection{优化算法} \label{subsec:minimize}
网格变形的过程是一个无约束非线性优化的过程。为了高效的优化,我们使用了\cite{Fu2015}中的非精确坐标块轮换下降算法,利用图着色算法进行并行加速。所以在更新一个顶点的时候,我们只需要实施一步梯度下降,计算效率比较高,同时使用回溯线搜索保证优化能量的下降和显式避免翻转四面体的出现。优化终止的条件是,能量收敛或者达到最大的迭代次数。和~\cite{Huang2014}相比,我们没有使用全局牛顿法类的优化算法,因此不需要计算整个能量函数的Hessian矩阵和通过求解线性系统来更新顶点,导致我们的算法的效率远远好于它。

\section{网格标记和压平} \label{sec:label_and_flatten}
和~\cite{Gregson2011,Livesu2013}的方法类似,在网格变形过程中,希望对整个网格表面三角形赋予正确的标记集而结束变形。在\ref{subsec:label}节中,讲明了如何对网格表面三角形进行标记,同时阐述了对于整体网格什么样的标记集是正确的。在拥有了正确的标记集后,在\ref{subsec:flatten}节中,首先鲁棒的计算每个表面三角形$\mf_i$上与对齐轴相容的目标位置分量,最后将网格压平,使表面严格的和轴对齐。

\subsection{标记} \label{subsec:label}
对于每个$\mf_i$,根据$\Ax(\mn_i)$的值,对变形后的网格表面进行标记,同时将拥有一样标记的相邻的三角形组成一个块$\mc_k$,这时能够得到初始的标记集$\mL_0$。设$\mc_k$的标记就是它里面的三角形的标记。$\mc_k$是多立方体结构的面,相邻的$\mc_i$ 和$\mc_j$ 的公共边是多立方结构的边$\me_k$,被多于2个$c_k$分享的原始表面网格顶点是多立方体结构的顶点$\mp_i$。不幸的是根据~\cite{Eppstein2010},对于标记好的表面网格,暂时没有找到充分必要的拓扑条件,来保证这个标记集肯定能导出多立方体结构。我们将~\cite{Eppstein2010} 中的准则作为充分条件来判断标记集的正确性,这和~\cite{Livesu2013}是类似的。准则如下:
\begin{itemize}
    \item 多立方体结构的任何一个面$\mc_k$的邻域数目不能少于4个;
    \item 多立方体结构的任何两个拥有相反朝向(比如+X与-X)的面不能共享一条边$\me_k$;
    \item 多立方体结构的任何顶点$\mp_i$的度为3。
\end{itemize}

在判断标记集$\mL_0$正确与否前,我们根据以下简单的准则对$\mL_0$进行如下操作。
\begin{enumerate}
\item 通过修改$\me_k$的邻域中三角形$\mf_i$的标记,尝试将每条$\me_k$拉直。如果$\mf_i$的三个相邻三角形的标记中有两个与$\mf_i$的不一样,就改变$\mf_i$的标记为那个不一样的标记。
\item 记$\mc_k$上的$\mf_i$的数目是$N_{\mc}$。如果$N_{\mc} < \epsilon_1$且$\mc_k$的邻域数目小于4,我们改变$\mc_k$的标记。如果$\mc_k$只有一个相邻块,那么它的标记直接改为相邻块的标记;相邻块的数目为2或者3的$\mc_k$,相邻块可以使用广度优先的算法将$\mc_k$的标记改变成自己的标记。如果$N_{\mc} \geq \epsilon_1$,那么直接判定$\mL_0$是不正确的。
\end{enumerate}

使用拉普拉斯光滑将每条$\me_k$进行平滑,将$\me_k$ 投影到它的期望方向上得到$\tilde{\me}_k$,$\tilde{\me}_k$经过$\me_k$的平均位置。如果$\tilde{\me}_k$存在拐点,那么它也是多立方体结构的极限点。$\me_k$的期望方向定义为与之相邻的$\mc_i$和$\mc_j$的标记对应的轴方向的叉积,叉积的顺序对寻找极限点没有影响。
一个极限点的严重程度$S$定义为一个三角形的面积,这个三角形的三个顶点是极限点本身,$\me_k$上离极限点最近的两个其他的极限点或者多立方体结构的顶点。

如果所有$S<\epsilon_2$且$N_{\mc} < \epsilon_1$,并且同时满足上面提到的三个准则,则这时的标记集是正确的。其中$\epsilon_1$和$\epsilon_2$是两个阈值,在我们的实现中,$\epsilon_1 = 5$,$\epsilon_2 = 3 \cdot \frac{1}{n} \sum_{i=1}^n \area(\mf_i)$。

\subsection{压平} \label{subsec:flatten}
虽然经过上面的操作,我们能够得到正确的标记集,但是这时网格表面还没有和轴严格的对齐。多立方体结构中每个$\mc_k$上表面网格顶点位置中有一个分量是一样的,并且这个分量是和$c_k$的标记相容的,设它的值为$v_k$。为了压平网格的表面,我们需要首先确定每个$\mc_k$上的$v_k$,这样就定义了最终的多立方体结构的表面位置约束,然后利用第\ref{chap:affine}章中保证无翻转的算法在满足约束的前提下,将网格变形成多立方体结构。当网格(二维或者三维)的边界映射是一一映射的时候,没有翻转的网格单元的映射是一个一一映射~\cite{Lipman2012,Aigerman2013}。所以在计算映射的时候,首先计算$\me_k$的映射(一维),其次是$\mc_k$的映射(二维),最后才是整个网格的映射(三维)。这样我们能够生成无翻转的映射,甚至是一一映射。最终的映射不是双射的原因是,网格不同部分可能存在相交,但是这种情况对于某些应用而言是无所谓的,比如六面体网格生成。

我们利用如下的二次规划算法寻找每个$\mc_k$的$v_k$:
\begin{equation} \label{equ:QP_flatten}
\begin{split}
    \min \   &\sum_{v_i} (v_i - m_i)^2 \\
      s.t. \ &v_j - v_k > l_{j,k},
\end{split}
\end{equation}
其中$m_i$是$\mc_i$上顶点坐标值中和轴相容的平均值,约束描述的是某些配对的$\mc_j$和$\mc_k$需要满足的要求。一旦优化问题\ref{equ:QP_flatten}没有解,我们会通过$l_{j,k} = 0.8 \cdot l_{j,k}$缩短$l_{j,k}$直至收敛。

为了寻找配对的块($\mc_j$,$\mc_k$),确定它们之间的前后顺序约束和距离约束$l_{j,k}$,我们使用下面的两个准则:
\begin{enumerate}
\item 多立方体结构的每条$\me_k$ 的端点被三个块共享,去除和$\me_k$相邻的两个块,两个端点处剩下的块构成配对的块。将两个端点在$\tilde{\me}_k$上的位置之差设为$l_{j,k}$,很容易能做到$l_{j,k}>0$,这样块之间的顺序约束也能确定;
\item 在每个块$\mc_i$上,寻找到平行的多立方体结构的边集合,两两组合成配对的边。如果可以使用多立方体结构的边将配对的边连接起来,我们去除这个配对的边;如果配对的边之间没有公共部分,也被去除掉。在配对的边($\me_j$,$\me_k$)相邻的块中,有一个块是一样的,就是$\mc_i$,将剩下两个不同的块组成配对的块($\mc_j$,$\mc_k$)。$l_{j,k}$通过$\tilde{\me}_j$和$\tilde{\me}_k$之间的公共区域的平均距离决定的,同样强制$l_{j,k}>0$。
\end{enumerate}
\textbf{讨论2:}\, 和~\cite{Gregson2011}类似,我们利用变形后的网格位置和正确的标记集来寻找配对的块,块之间的前后顺序和距离约束。但是和~\cite{Gregson2011}的不同点是,我们将这些约束设为硬约束,能够保证生成合理的$v_k$,然而~\cite{Gregson2011}只是将它们作为软约束,不能保证之前定义的顺序和距离约束。

对于六面体网格生成的目标,$v_i$应该是目标方格长度(用户定义)的整数倍。所以我们会将优化问题\ref{equ:QP_flatten}中的$m_i$四舍五入成方格长度的整数倍。不等式的右端被设成$l_{j,k}$和方格长度的最大值,用来避免产生退化的四面体。\ref{equ:QP_flatten} 优化结束后,我们将$v_i$ 四舍五入成方格长度的整数倍。因为使用了硬约束的优化,我们能鲁棒的产生具有不同方格长度的六面体网格,如图\ref{fig:diff_resolution_hex}所示。

\section{实验与比较} \label{sec:pc_results}
我们的实验是在一个拥有英特尔3.4\,GHz CPU和16\,GB RAM的台式机上运行。为了产生无翻转的结果,输入的四面体网格需要满足,没有任何内部边或者面的所有顶点都在边界上\cite{Aigerman2013}。在我们的实现中,我们将这些边和面劈开,生成合理的输入网格。使用和第\ref{chap:AMIPS},\ref{chap:affine}章一样的形变定义来度量映射的好坏,主要是刚性形变。

\subsection{无翻转的四面体}
在后续的应用中,多立方体结构中翻转的四面体会导致其他算法的失败。比如在六面体网格生成中,如果一个六面体网格单元的顶点在一个翻转的四面体中,它反投影回原始网格中的位置是不明确的,存在二义性。图\ref{fig:flipped_cmp}中给出了和~\cite{Huang2014} 的比较,我们算法能生成无翻转,并且低形变的多立方体结构。在~\cite{Huang2014}和~\cite{Gregson2011}中,使用的能量是基于四面体与顶点的尽量刚性(ARAP)能量,它们都没有能力保证没有翻转的单元出现,比如~\cite{Gregson2011} 中的图9。

\begin{figure}[t]
\centerline
{
\begin{overpic}[width=0.99\columnwidth]{Polycube/flipped_cmp}
\put(15,-3){\textbf{(a)}}
\put(50,-3){\textbf{(b)}}
\put(88,-3){\textbf{(c)}}
\end{overpic}
}
\vspace{4mm}
\caption{多立方体映射中有无翻转的比较。(a) 原始的大象模型。(b)~\cite{Huang2014}的方法生成的结果,存在592个翻转的四面体,使用黄色表示。(c)我们的算法生成的无翻转四面体的多立方体结构,它最大的刚性形变是10.46。}
\label{fig:flipped_cmp}
\vspace{-3mm}
\end{figure}

在优化问题\ref{equ:QP_flatten}中,我们使用了硬约束来表达不同块之间的前后顺序和距离约束,所以$v_k$的求解是非常鲁棒的。如果不使用我们的算法,两个相邻平行的块($\mc_i$,$\mc_j$),四舍五入它们的$v_i$和$v_j$可能会得到同一个值上,而导致出现退化的四面体。针对六面体网格生成,我们可以提供具有不同方格长度的六面体结果(不是简单的细分后的不同分辨率)。图\ref{fig:diff_resolution_hex}展示了同一个模型上不同分辨率的六面体网格。目标方格长度分别为输入四面体网格表面边长的平均值的0.8倍(图\ref{fig:diff_resolution_hex}(b))和2.5倍(图\ref{fig:diff_resolution_hex}(c)),我们的方法都能成功的生成六面体网格。六面体网格的最小和平均缩放Jacobian行列式值分别为(0.410, 0.920)(图\ref{fig:diff_resolution_hex}(b))和(0.314, 0.822)(图\ref{fig:diff_resolution_hex}(c))。可以看到增加六面体网格的单元数量,可以在某些情况下提高整体网格的质量。图\ref{fig:diff_resolution_hex}(e) 和(f)表示的奇异性结构是一样的,也就是说这两个不同分辨率的六面体网格是来自同一多立方体结构。
\begin{figure}[t]
\centerline
{
\begin{overpic}[width=0.99\columnwidth]{Polycube/diff_resolution_hex}
\put(15,-3){\textbf{(d)}}
\put(50,-3){\textbf{(e)}}
\put(88,-3){\textbf{(f)}}
\put(15,32){\textbf{(a)}}
\put(50,32){\textbf{(b)}}
\put(88,32){\textbf{(c)}}
\end{overpic}
}
\vspace{4mm}
\caption{同一个模型产生不同分辨率的六面体网格。(a) 原始的Fertility模型。(d)我们的算法生成的无翻转四面体的多立方体结构。(b)和(c)是两个不同分辨率的六面体网格,分别拥有34302和2709个六面体。(e)和(f)分别是(b) 和(c) 的奇异性结构。因为不同的目标方格目标长度会要求两个多立方体结构,这里(d)只是展示了一个多立方体结构的作为代表,用于生成(b)。}
\label{fig:diff_resolution_hex}
\vspace{-3mm}
\end{figure}

\subsection{高效性}
在构造多立方体结构时,算法的效率是非常重要的,如果自动化的算法效率比较低,还不如直接使用人工进行设计。图\ref{fig:time_L1_cmp}展示了和~\cite{Huang2014}在同一个Kiss模型上构造多立方体结构的效率比较。图\ref{fig:flipped_cmp}(b)和(c)花销时间分别为23.4分钟,13.05秒。在这两个例子中,和~\cite{Huang2014}相比,我们都使用了接近一半的四面体数目的网格作为输入,虽然这样看上去有点不公平,但是我们的非精确的坐标轮换下降算法基本上是和四面体数目成正比的,也就是我们算法的时间大概也就是现在的两倍左右,这还是远低于~\cite{Huang2014}的时间。例外,我们根据~\cite{Livesu2013}的Kiss模型表面三角形网格生成四面体网格作为输入,共有442736个四面体,我们的算法大概花费了2.5分钟构造多立方体结构,1.5 分钟严格压平表面网格,其实我们的算法不需要这么多的四面体也能获得很好的结果。
我们的算法高效的原因如下,使用了高效的非精确坐标块轮换下降算法变形网格,同时不要求在开始的网格变形中严格压平所有的表面三角形而减少了迭代次数。而~\cite{Huang2014}使用了类似牛顿法进行优化,那么建立Hessian矩阵和求解线性系统的效率都很低。~\cite{Livesu2013}使用了启发式的局部搜索的策略消除极限点,同样效率比较低。

\begin{figure}[t]
\centerline
{
\begin{overpic}[width=0.99\columnwidth]{Polycube/time_L1_cmp}
\put(10,-3){\textbf{(a)}}
\put(35,-3){\textbf{(b)}}
\put(60,-3){\textbf{(c)}}
\put(90,-3){\textbf{(d)}}
\end{overpic}
}
\vspace{4mm}
\caption{Kiss模型上构造多立方体结构的效率比较。(a),(b)分别是~\cite{Huang2014}花了大约37.3分钟构造出的多立方结构,和对应的六面体网格。(a)图上平面上出现的瑕疵是翻转的四面体导致的。(c),(d)分别是我们的算法使用15.55秒生成的无翻转四面体的多立方体结构和对应的高质量六面体网格。}
\label{fig:time_L1_cmp}
\vspace{-3mm}
\end{figure}

\subsection{可控的奇异性}
用户可以根据某个参数的调节生成具有不同奇异性的多立方体结构。这种控制在某些应用是非常重要的,比如应用中奇异点数目和映射形变是相互矛盾的。我们的方法通过高斯函数的核宽度$\sigma_s$来控制奇异性。
图\ref{fig:diff_sigma_cmp}展示了在同一模型上,使用不同$\sigma_s$值来生成多立方体结构。从映射形变和奇异点的数目来看,大的$\sigma_s$可以生成更少的奇异点数目,更大的形变;小的$\sigma_s$能产生更多的奇异点数目,更少的形变。

\begin{figure}[t]
\centerline
{
\begin{overpic}[width=0.99\columnwidth]{Polycube/diff_sigma_cmp}
\put(10,-3){\textbf{(a)}}
\put(35,-3){\textbf{(b)}}
\put(60,-3){\textbf{(c)}}
\put(90,-3){\textbf{(d)}}
\end{overpic}
}
\vspace{4mm}
\caption{Buste模型上使用不同的$\sigma_s$来构造具有不同奇异性的多立方体结构。(a)原始Buste模型。(b),(c),(d)分别使用1,1.5,2.0倍平均输入表面网格的平均边长作为$\sigma_s$,分别生成具有128,72,36个奇异点的多立方体结构,平均刚性形变分别为1.50,1.53,1.57。}
\label{fig:diff_sigma_cmp}
\vspace{-3mm}
\end{figure}

\begin{figure}[t]
\centerline
{
\begin{overpic}[width=0.8\columnwidth]{Polycube/fertility_hex_cmp}
\put(20,50){\textbf{(a)}}
\put(62,50){\textbf{(b)}}
\put(20,-3){\textbf{(c)}}
\put(62,-3){\textbf{(d)}}
\end{overpic}
}
\vspace{4mm}
\caption{Fertity模型上使用不同的多立方体结构构造算法生成的六面体网格。(a),(b),(c),(d)分别来自\cite{Gregson2011},~\cite{Huang2014},~\cite{Livesu2013},我们的方法。六面体的数目分别是19870,53702,17910,24920。缩放Jacobian行列式的最小和平均值分别为(0.196,0.911),(0.260,0.872),(0.312,0.911),(0.422,0.917)。}
\label{fig:fertility_hex_cmp}
\vspace{-3mm}
\end{figure}

\subsection{六面体网格生成}
为了生成六面体网格,我们在生成多立方体结构的时候,已经将$v_k$都四舍五入成目标方格宽度的整数倍,因此可以直接将模型按照方格宽度进行缩放,使得所有的$v_k$变成整数。然后直接在整数点上直接建立六面体网格。最后将六面体网格的顶点根据它在四面体中的重心坐标反投影回输入网格区域,这样就得到了输入网格的六面体化。尽管在优化中,我们保证了无翻转,但是这只能保证每个六面体网格的顶点在一个四面体内,不能保证反投影后的六面体网格的缩放Jacobian行列式都大于0。 因此为了获得高质量的六面体网格,我们提出了如下的优化方法。

设反投影后的六面体网格为$\mM^H$。 首先在$\mM^H$的表面上插入一层~\cite{marechal2009,shepherd2007,Gregson2011}。为了去除拥有负的缩放Jacobian行列式的六面体(称为翻转的六面体),我们首先设计了一个简单的能量函数,通过优化它,能够生成无翻转的六面体。将一个六面体分成8个四面体,每个四面体的四个顶点是六面体的一个顶点$\mv_0^H$,在六面体上与$\mv_0^H$相邻的三个顶点$\mv_1^H$,$\mv_2^H$,$\mv_3^H$。对于每个四面体$\triangle \mv_0^H\mv_1^H\mv_2^H\mv_3^H$而言,将$\delta_{conf}$ 中的$\det(A)$变成$0.5 \cdot (\det(A) + \sqrt{\det(A)^2 + \zeta})$来定义新的能量,其中$\zeta$是一个比较小的正数,取法见~\cite{Escobar2003}。因此对于六面体网格而言,就是将8 个四面体上的能量相交。使用非精确坐标块轮换下降算法进行优化,效率很高。这个思想是和~\cite{Escobar2003,Sastry2014}类似的。虽然不能保证最后一定没有翻转的六面体,但是在实际中的效果很好。一旦没有翻转的六面体后,使用AMIPS~\cite{Fu2015}能量继续优化。为了获得更好的表面四边形,将\cite{Fu2014}中的LCT能量推广到四边形上,用于生成各向同性的表面四边形。通过交替表面四边形网格优化和内部顶点优化,能够获得很高的六面体网格质量。

图\ref{fig:fertility_hex_cmp}给出了一组都是使用多立方体结构来生成的六面体网格的比较。不管从缩放Jacobian行列式的数值统计,还是六面体网格表面上的颜色来看,我们的方法都能产生最好质量的六面体网格。图\ref{fig:LY_hex_cmp}比较了我们的六面体网格和~\cite{Li2012}的结果。图\ref{fig:LY_hex_cmp}(a)(b)的缩放Jacobian行列式的最小,平均值,六面体数目分别为:(0.293,0.940,133632), (0.336,0.947,59841);图\ref{fig:LY_hex_cmp}(c)(d)分别为(0.209,0.866,10600), (0.454,0.907,14606),所以我们的结果要好于~\cite{Li2012},同时图上六面体的颜色也反映了这个结论。

\begin{figure}[t]
\centerline
{
\begin{overpic}[width=0.8\columnwidth]{Polycube/LY_hex_cmp}
\put(24,40){\textbf{(a)}}
\put(70,40){\textbf{(b)}}
\put(24,-3){\textbf{(c)}}
\put(70,-3){\textbf{(d)}}
\end{overpic}
}
\vspace{4mm}
\caption{和~\cite{Li2012}比较生成的六面体网格。(a),(c)是~\cite{Li2012}的结果。(b),(d)是我们的结果。}
\label{fig:LY_hex_cmp}
\vspace{-3mm}
\end{figure}


\section{本章小结}
我们提出了一个自动构造多立方体结构的算法,首先通过网格变形找到正确的标记集,然后通过二次规划找到多立方体结构边界面的目标位置值,最后使用现有的保证无翻转的算法生成最终的多立方体结构。高斯平滑的核宽度是一个可调的参数,用来平衡映射的形变和多立方体结构的奇异性。通过将我们的算法应用一些模型上和进行六面体网格生成应用上,证明了算法能保证无翻转,高效性,奇异性可控等优点。同样,我们有一些局限性,需要在未来的工作中去解决。

\textbf{理论保证。} 对于任意输入的网格和任意的高斯平滑的核宽度,我们不能保证一定成功的生成合理的多立方体结构。对于寻找正确的标记集,没有理论保证能一定成功,因为我们使用了非线性优化算法来变形网格。在生成最终的多立方体结构的时候,因为使用的寻找无翻转的映射的技术,不能保证成功,导致我们的算法没有理论保证。

\textbf{高斯函数的核宽度}用来平衡映射形变和多立方体结构的奇异性。暂时,用户需要通过不停的尝试,才能找到一个最优的宽度。比如一些用户认为有兴趣的特征,相对与核宽度是很小的,导致最终的多立方体结构上不能体现。所以根据感兴趣的特征,设计一个空间变化的核宽度是非常有必要的。但是如何定义特征的尺度是一个非常有挑战的问题。我们想在未来进行这方面的尝试。

\textbf{多立方体结构的拓扑条件。} 我们使用的充分拓扑条件是一个强充分条件,暂时还没有弱充分条件出现。\cite{Huang2014}的结果中有度不是3的多立方体结构顶点产生,他们依然能产生正确的结构。同样在我们的实验中,如果将两个非常近的度为3的顶点合并成一个度为4的顶点也是可以的,注意为了最后产生无翻转的结果,度为4的多立方体结构顶点在原始网格表面上的度至少为6。所以在未来的工作中,寻找多立方体结构的充分必要拓扑条件是一个有趣课题。
