\begin{cnabstract}
在科学研究、工程计算、文化娱乐中,数字几何数据扮演着越来越重要的角色。使用数学模型和算法来分析与处理数字几何数据的过程称作数字几何处理。这是一个包含计算机科学、应用数学和工程学等学科的交叉性研究课题。常见的研究内容包括模型获取、模型重建、网格生成、形状分析与理解、映射计算和几何建模等。我们的研究针对数字几何处理中的两个子课题:最优映射计算和最优网格生成。其中最优映射计算是一个重要的课题,它是许多计算机图形学应用的核心,比如网格参数化、网格变形、网格质量提高、六面体网格生成。最优网格生成是网格数据处理的基石,比如各向异性的网格和六面体网格有很强的需求,因为在有限元方法中,它们比各向同性网格和四面体网格能获得更好的计算精度。

本文从优化的角度设计了新颖的能量函数和优化方法,将它们成功地应用到了最优网格映射计算、各向异性网格生成和多立方体结构(polycube)自动生成这几个课题,具体如下:

一个好的映射算法需要保证无翻转、低形变和计算高效性。现有的算法不能同时保证这些特性。本文设计了一个新颖的形变最小化能量(AMIPS),并使用非精确块坐标轮换下降算法来快速地计算无网格翻转的最优映射。其中AMIPS能量函数继承了传统的MIPS能量的保证无翻转的性质,同时能控制最大的形变。非精确块坐标轮换下降算法(inexact BCD)能避免优化过程过早地陷入局部最小。结合AMIPS能量函数与inexact BCD优化算法,本文提高了映射的计算效率和质量。在网格参数化、二维三角形网格与三维四面体网格变形、二维与三维无网格变形、各向异性四面体和六面体网格质量提高等应用中充分体现了我们算法的优越性。

但是AMIPS算法同样存在缺点:比如不能支持存在很多控制点的网格变形,而且对初始映射比较敏感。本文提出了一个网格组装分离方法来计算无翻转的最优映射。我们的方法接受任意的网格映射作为输入,该输入映射可以存在众多翻转的网格单元。我们首先将网格的所有网格单元分离,保持每个网格单元上的映射是低形变的,然后通过同时优化形变和分离顶点之间的距离来计算无翻转的最优映射。由于使用了每个网格单元上的仿射变换作为优化变量,我们可以通过求解一个无约束的非线性非凸优化问题来得到最优映射。同样在平面网格参数化、网格变形等应用中体现了我们算法的鲁棒性和高效性。

在几何建模、物理模拟和机械工程等应用中,各向异性网格是非常重要的。本文提出了局部凸函数三角化(LCT)方法,用于生成高质量的各向异性网格。 输入一个曲面,或者一个三维空间区域作为定义域,和在定义域上的已知黎曼度量场,我们将各向异性网格生成问题转化为一个函数逼近问题。在每个网格单元上构造局部凸函数,它的Hessian矩阵局部上和输入的黎曼度量一致,我们利用网格顶点移动和改变网格连接关系的策略来降低函数逼近误差。我们的LCT方法推广了最优Dealunay三角化(ODT),可以接受黎曼度量场作为输入,同时可以适用于剧烈变化的黎曼度量场和存在尖锐特征的网格。在二维平面区域、三维空间区域和三维曲面上的各向异性网格生成应用中很好了体现我们算法的高效率和高质量的优点。

在物理模拟和机械工程等应用中,六面体网格往往比四面体网格有着较好的性质,比如更少的网格单元、更高的计算精度。本文通过高质量多立方体(PolyCube)结构来生成六面体网格。多立方体结构要求网格的表面三角形的法向和X,Y,Z轴严格对齐。之前的算法不能同时保证无翻转、低形变、奇异性可控和计算高效。本文使用非精确块坐标轮换下降算法来优化表面法向光滑与对齐能量,用来驱动网格变形并自动地消除极限点,以自动生成高质量的多立方体结构。我们引入光滑函数的核宽度来控制多立方体结构的奇异性。非精确块坐标轮换下降算法的高效率使本文的自动化算法的效率远远高于现在最先进的算法。从多立方体映射的形变和六面体网格生成的结果来看,我们算法的质量和效率相对于当前最先进的算法都有较大提升。


\keywords{数字几何处理\enskip 各向异性网格生成\enskip 局部凸函数网格化\enskip 无翻转映射\enskip 多立方体结构\enskip 形体变形 \enskip 网格参数化}
\end{cnabstract}

\begin{enabstract}
3D digital contents play important roles in the scientific research, mechanical engineering and entertainment. Digital geometric processing is a field chat is concerned with mathematical models and algorithms for analysing and manipulating geometric data. It is a interdisciplinary research relating to computer graphics, applied mathematics and engineering. Typical operations include mesh acquisition, mesh reconstruction, mesh generation, shape analysis and understanding, mapping computation and geometric modeling. Our research focuses on two topics: optimal mapping computation and mesh generation. Optimal mapping computation is a fundamental task in computer graphics, and it is the key in many applications, e.g. mesh parameterizations, mesh deformation, mesh improvement, all-hex mesh generation. Optimal mesh generation is the cornerstone of digital geometric processing. For example, anisotropic and all-hex meshing are in great need, because they can provide more accuracy than isotropic and tetrahedral mesh in FEM. 

This paper designs novel energy functions and optimization algorithms from the perspective of optimal theory. We apply them on optimal mapping computation, anisotropic mesh generation and automatic PolyCube structure construction:

Optimal mappings need no-inversion elements, low distortion and efficiency. State-of-the-art methods can't satisfy all requirements.By revisiting the well-known MIPS (Most-Isometric ParameterizationS) method, we introduce an advanced MIPS(AMIPS) method that inherits the local injectivity of MIPS, achieves as low as possible distortions compared to the state-of-the-art locally injective mapping techniques, and performs one to two orders of magnitude faster in computing a mesh-based mapping. The success of our method relies on two key components. The first one is an enhanced MIPS energy function that penalizes the maximal distortion significantly and distributes the distortion evenly over the domain for both mesh-based and meshless mappings. The second is a use of the inexact block coordinate descent method in mesh-based mapping in a way that efficiently minimizes the distortion with the capability not to be trapped early by the local minimum. We demonstrate the capability and superiority of our method in various applications including mesh parameterizations, mesh-based and meshless deformation, and mesh improvement.

AMIPS still has some limitations: it can't support many handles in mesh deformation and is sensitive to initial mapping. We present a novel method to compute locally injective mappings with low distortion on simplicial meshes.  Given an initial mapping with or without inverted simplices, our method first disassembles it into disjointed and inversion-free simplices by modifying the piecewise affine transformations defined on simplices, then minimizes the mapping distortion and the difference of the disjointed vertices with respect to the piecewise affine transformations. Due to the use of transformations as unknowns, our algorithm explicitly guarantees the local injectivity of the mapping via an unconstrained minimization. Compared with existing methods, our method is robust to initial mappings even with many inverted elements and handle constraints. Our method is also capable of achieving bounded distortion mappings. We demonstrate the efficiency and robustness of our method on a variety of applications.

Anisotropic mesh is very important in geometric modeling, physical simulation and mechanical engineering. We present a novel approach for high-quality anisotropic triangle mesh generation, called local convex triangulation(LCT).
Given a 2D or surface domain equipped with Riemannian metrics, the anisotropic meshing is transformed into a functional approximation problem. We construct convex functions locally over the mesh to best match Riemannian metrics, and adapt vertex positions and mesh connectivity to minimize the interpolation error iteratively to achieve the desired anisotropy. We show that our method is a generalization of Optimal Delaunay Triangulation and we develop a simple and efficient algorithm that works well for 2D and surface meshing. The superiority of our method in mesh quality and algorithmic efficiency in comparison to existing methods is demonstrated with a variety of models and Riemannian metrics.

All-hex meshes possess nice numerical properties, such as a reduced number of elements and high approximation accuracy in physical simulation and mechanical engineering. This paper generates high-quality all-hex mesh based on PoilyCubes which are good abstractions of closed shapes. PolyCubes need the normals of boundary faces are aligned with X, Y, Z axis rigorously. A desired PolyCube construction method should (1) provide a low-distortion map without foldover and degeneracy; (2) offer flexible control on singularity counts; (3) compute the result efficiently and automatically. We introduce a novel method that fulfills these requirements to compute PolyCubes for tetrahedral meshes. We regard the computation as mesh deformation that is driven by face normal smoothing and axis-directional alignment under distortion control.
The kernel size of the smoothing is used to control the singularity count and the level of alignment is adjusted automatically to resolve the turning point issue.  We formulate the deformation as an optimization problem and propose a very efficient solver.
We demonstrate the quality and speed of our method compared to state-of-the-art methods on a variety of models.

\enkeywords{Digital geometric processing, Anisotropic, Local convex triangulation, Inversion-free mappings, PolyCube, Parameterizations, Deformation }
\end{enabstract}
