\chapter{计算原始区域四边形网格}
%利用第\ref{chap:maquad}章的算法,我们保持原始区域上的三角网格$\mM^p$的连接关系,把每个点$\mvar$映射到$\mvar^'$,就得到多正多边形结构的网格$\M '$。提取$\M '$的边界,再利用第\

本章目标是借助点在三角形中的重心坐标,把多正多边形结构上生成的四边形网格映回原始区域。记原始三角网格为$\mM^0$,其点集合为$\mV^0=\{\mvar_1^0,...,\mvar_n^0\}$,面集合为$\mF^0=\{\mf_1^0,...,\mf_m^0\}$,对应的多正多边形结构的三角网格为$\mM $,其点集合为$\mV=\{\mvar_1,...,\mvar_n\}$,面集合为$\mF=\{\mf_1,...,\mf_m\}$,再$\mM$上生成的四边形网格点集合为$\mQ={\mq_1,...\mq_l}$。

由于任意的四边形网格点必落在$\mM$的某个三角形的边界或内部,我们可以用这个三角形的中心坐标表达这个点,即

$\,\,\,\,\,\,\,\forall \mq_i \in \mQ, \exists ! \{\mvar_r,\mvar_s,\mvar_t\}=\mf_j \in \mF, s.t.$
\begin{equation}\label{equ:meancoor}
\mq_i=r\mvar_r+s\mvar_s+t\mvar_t\,,r+s+t=1\,,r,s,t\geq 0
\end{equation}
由于上式也是点落在三角形边界或内部的充分条件,故我们直接把重心坐标应用于原网格上的三角形,即计算出
\begin{equation} \label{equ:quadvertex}
\mq_i^0=r\mvar_r^0+s\mvar_s^0+t\mvar_t^0
\end{equation}
其中$\mq_i^0$是原始区域上的四边形网格的点,它也落在相应的三角形上,四边形网格的连接关系不变。由此可以看出,若第\ref{chap:polyquad}章计算的映射形变较大,在原始区域上生成的四边形网格质量就会变差。最后我们可以利用第\ref{chap:sus2d}章的方法提高网格质量。图给出了由重心坐标得到的四边形网格及优化后的结果。

%\emph{注}\,如果需要生成的四边形网格边界尽量接近原始的边界,可以先在多正多边形结构上对四边形网格进行加密,也可以把差距较大的边界边细分,并插入原始边界的顶点。图中
