\chapter{总结与展望} \label{chap:conclusion}
在科学研究、工程计算、文化娱乐中,三维数字模型扮演着越来越重要的角色。随着三维扫描技术和软件工具的发展,我们已经能很容易的获得这些三维数字模型。但是为了后续的应用,原始的数据一般不能被直接使用,需要进行相关的分析和处理,这个过程叫做\emph{数字几何处理}。本文研究的课题就是数字几何处理中的两个子课题,包括\emph{网格生成}和\emph{映射计算}。

\section{本文工作总结}
映射计算是计算机图形学中重要的基础课题,它可以被广泛地应用,比如平面网格参数化、网格变形、网格质量提高。\emph{最优映射}具有无翻转,低形变,计算效率高的性质。为了得到这样的映射,我们首先提出了AMIPS技术,它扩展了著名的MIPS 方法。AMIPS继承了MIPS保证无翻转性质,同时能显著地控制了最大的形变,显著地提高了计算效率。AMIPS的核心想法是,使用指数函数与MIPS能量结合,用于控制最大形变;使用非精确块坐标轮换下降算法进行优化,避免优化算法过早的陷入局部极小。但是AMIPS也存在局限性,比如不能处理带有大量控制点的网格变形,于是我们提出了基于组装分离的网格单元的最优映射计算,它首先能满足上述最优映射的要求,并且对初始映射,控制点的数目不敏感,同时可以计算形变有界的映射。它的基本出发点是将输入网格分离成不相连的网格单元,这时每个网格单元都是不翻转的,以每个网格单元上的仿射变换作为变量,建立一个无约束优化问题来组装分离的网格单元。在优化的过程中,保证网格单元上的映射一直满足无翻转、低形变的要求。在平面网格参数化、网格变形、网格质量提高等应用上的实验结果,相比于最先进算法我们算法具备很强的优越性。

各向异性网格在几何建模、物理模拟、机械工程等应用中,具有非常广泛的用途,可以提高数值模拟的精度。为了使网格生成算法能够适用于,一般化的黎曼度量场,同时适应各向异性变化剧烈的黎曼度量场和定义域网格中存在尖锐特征,我们提出了局部凸函数三角化(LCT)。LCT扩展了最优Delaunay三角化方法。LCT的关键思想是在每个网格单元上构造局部凸函数,它的Hessian矩阵局部上和输入的黎曼度量一致,使用网格顶点移动和改变网格连接关系的方法去降低函数逼近误差,生成满足输入的网格。在二维平面区域、三维曲面区域和三维体区域上生成的高质量各向异性网格证明了我们的算法能提供极高的计算效率和网格质量。

多立方体结构是一种特殊的网格结构,网格表面三角形的法向是($(\pm 1,0,0)^T$, $(0,\pm 1,0)^T$ 或者 $(0,0,\pm 1)^T$)中的一个。多立方体结构能够被广泛地应用,比如六面体、四边形网格生成,纹理映射等。高质量的多立方体构造算法需要是自动的,映射无翻转,低形变,奇异性可控,算法效率高。我们提出了一种新颖的基于网格变形的算法,利用网格表面法向光滑与对齐的能量快速的构造多立方体结构。算法中的高斯函数的核宽度是一个可调的参数,用来平衡映射的形变和多立方体结构的奇异性。我们的算法应用到六面体网格生成中能生成无翻转,计算效率高,奇异性可控的多立方体结构。我们同时提出的六面体网格优化算法也可以显著地提高六面体网格的质量。

\section{未来工作展望}
虽然我们的算法都能够生成高质量的各向异性网格、低形变的映射和高质量的多立方体结构,但是它们都存在一个缺点,没有严格的理论保证,比如生成的各向异性网格没有质量的保证,最优映射的计算不能保证一定成功,多立方体结构的构造不能保证收敛。因此在接下来的研究中,我们希望能够从理论出发,在保证整体质量的前提下,开发出具备理论保证的算法。

最优映射的计算效率虽然现在已经很高,但是和基于线性的方法比较,还是比较慢。我们希望在未来的工作中能设计计算更快,形变更低的映射,比如对于大尺度的网格变形能够获得交互级别的反馈。另外双射在网格映射中具有很重要的地位,比如在纹理映射中,它能够提供图像和网格之间的一一映射。我们现在的算法主要针对局部单射的映射,如何计算有效的双射也是未来的研究方向。

因为四边形和六面体网格在有限元方法等应用中,具备计算精度高的优点,因此它们需要更加好的算法来生成。现在我们各向异性网格生成算法只能应用在三角形和四面体网格上,同时现在也存在很多的算法用来生成各向同性的四边形和六面体网格,所以我们希望在未来的工作去设计算法,将各向异性的概念推广到四边形或者六面体网格。

虽然现在我们提出了很高效的多立方体构造算法,用于生成六面体网格,但是最终的六面体网格在内部没有奇异性,导致奇异结构的形式比较单一,最终使六面体网格质量不够好。所以我们希望在构造好的多立方体结构上直接设计奇异结构来辅助生成更高质量的六面体网格。

最后,我们认为最优映射能够被更多的问题利用,不光是我们文中提到的问题。希望能有更多的问题可以利用最优映射的概念来实现以往不能实现的目标,使之在计算机图形学中能够广为流传。
