\chapter{绪论} \label{chap:introduction}
随着计算机科学的迅速发展,三维数字模型在众多的科学学科中起着越来越重要的作用。比如在神经科学、机械工程学与天体物理学中,科学家们可以通过数字模型的分析与处理技术来理解新的几何结构和发现新的科学规律。在科学计算领域中,经常需要求解各类偏微分方程,但是很多方程的解析解是很难得到的,有限元方法使用离散化的三维数字模型将连续区域转化为离散区域,在离散区域上利用数值技术去求解偏微分方程的数值解,可以得到近似的精确解。在娱乐(电影,视频游戏)、文物保护和建筑学等其他行业中,大量的三维数字模型被用来表达数字化的人物,建筑等,从而使得人们可以欣赏到栩栩如生的成绩,巍峨的建筑,并还原悠久的历史文化。众多数字动画影片中的角色都使用离散的三维数字模型来表示,如图\ref{fig:big_buck_bunny_model}所示,角色需要的动作可以通过操作这些模型来表达;文物的三维数字化能全方位、多角度地真实反映实体文物的细节如图\ref{fig:relic}所示,在后续的工作中可以减少实体文物被使用的次数,从而达到文物保护的目的;同时为实体文物保存了一份完整的数字数据,在实体文物意外受损时能根据这些数字模型进行修复,研究人员也能随时随地地使用数字模型进行研究。现在随着材料技术和三维打印技术的发展,上述行业中的很多三维数字模型可以被打印出来,被广泛用于科学研究、工程建造、文物保护、娱乐等。图\ref{fig:3D_printing}展示了将一个动画电影中的三维数字模型打印出来的例子。%类似这种个性化的三维打印技术,可以用来极大地丰富大众的娱乐,制造定制的模具等。

\begin{figure}[h]
\centerline
{
\begin{overpic}[width=0.95\columnwidth]{introduction/big_buck_bunny}
\end{overpic}
}
\vspace{1mm}
\caption{使用一个四边形网格表达电影《大雄兔》中的主角。右图展示了动画中模型的手臂是如何扭转的。图片来自\href{http://vcg.isti.cnr.it/Publications/2013/BLPPSTZ13a/big\_buck\_bunny.png}{http://vcg.isti.cnr.it/Publications/2013/BLPPSTZ13a/big\_buck\_bunny.png} }
\label{fig:big_buck_bunny_model}
\vspace{-2mm}
\end{figure}

三维数字模型在上述应用起到了非常重要的作用,人们希望能够更加容易地获取高质量的数字模型。现代快速发展的三维扫描设备,比如激光扫描仪、微软公司的Kinect设备、英特尔公司的RealSense摄像头等,使得我们能更容易地、更准确地、更廉价地获取真实世界中物体的三维数字模型。同样众多成熟的三维建模软件,如Autodesk公司的3Ds Max、AutoCAD、Maya与Robert McNeel公司的Rhino等,都可以用来设计、创造三维数字模型。这些模型可能是真实世界中不存在的,更加体现了人类非凡的创造力、想象力。通过这些技术得到的三维数字模型现在已能很有效地去服务很多行业。

\begin{figure}[t]
\centerline
{
\begin{overpic}[width=0.9\columnwidth]{introduction/relic}
\end{overpic}
}
\vspace{1mm}
\caption{技术人员在对洛阳博物馆的镇馆之宝东汉石辟邪进行三维扫描。图片来自 \href{http://www.kaogu.cn/cn/gonggongkaogu/2015/0608/50496.html}{http://www.kaogu.cn/cn/gonggongkaogu/2015/0608/50496.html}。}
\label{fig:relic}
\vspace{-1mm}
\end{figure}

\begin{figure}[t]
\centerline
{
\begin{overpic}[width=0.8\columnwidth]{introduction/3D_printing}
\end{overpic}
}
\vspace{1mm}
\caption{三维打印技术制造的电影《小黄人大眼萌》中的虚拟人物。图片来自 \href{http://yun.3deazer.com/archivecontent.yz?archiveId=62024}{http://yun.3deazer.com/archivecontent.yz?archiveId=62024}。}
\label{fig:3D_printing}
\vspace{-4mm}
\end{figure}

通过三维扫描技术获得的离散几何模型,通常含有很多的噪声,或者初始的表达方式不能适用于后续的应用,所以这样的模型需要经过一系列的后处理与优化,才能被真正的使用。从基本模型数据(点云,CT图像等)获取到可以被其他行业使用,这中间有一个\emph{数字几何处理}的过程存在,用来分析和操作获得的基本模型数据。本文研究的重点就是\emph{数字几何处理}中的几个子课题。 从\emph{数字几何处理}的表面文字上看,“数字”代表了处理的对象是存储在计算机中的,所以它和计算机科学(计算机图形学)是紧密相连的;“数字几何”表明了该领域的研究是以离散模型数据对象,以离散微分几何为理论基础;“处理”代表了在分析数据时需要一些算法,结果需要达到最优解,因此它以最优化理论,技术为理论基础;所以这是一个结合了计算机科学、离散微分几何与最优化理论的交叉学科。\emph{数字几何处理}就是将相应的几何处理算法应用到输入的几何模型数据上。其中算法代表的是\emph{操作},几何模型代表的是\emph{物体}。一般的\emph{操作}包括曲面重建、模型光滑、模型参数化、模型重新网格化与几何建模等,文献\cite{Botsch2010}对这些技术给出了详细的参考。\ref{sec:DGP}节介绍了本文涉及到的具体问题。在计算机图形学中,一般使用点云或者多边形网格来近似连续的\emph{物体}。其中\emph{多边形网格}已经成为最常用的表示方式,它可以通过对输入物体的点云实施曲面重建技术而获得,\ref{sec:representation}节讨论了这种基于多边形(多面体)的表达和本文处理的网格类型。

\section{三维数字模型的离散表达}\label{sec:representation}
\begin{figure}[t]
\centerline
{
\begin{overpic}[width=0.95\columnwidth]{introduction/mesh_graph}
\put(13,0){\textbf{(a)}}
\put(45,0){\textbf{(b)}}
\put(77,0){\textbf{(c)}}
\end{overpic}
}
%\vspace{1mm}
\caption{不同的离散几何物体。 (a)二维三角形网格。 (b)三维三角形曲面网格。 (c)三维四面体网格。}
\label{fig:mesh_graph_cmp}
\vspace{-3mm}
\end{figure}

\begin{figure}[t]
\centerline
{
\begin{overpic}[width=0.95\columnwidth]{introduction/tri_quad}
\put(22,0){\textbf{(a)}}
\put(71,0){\textbf{(b)}}
\end{overpic}
}
\vspace{1mm}
\caption{使用不同的多边形表示同样的曲面模型。 (a)只用三角形组成了离散曲面,称为三角形网格。 (b)网格上的多边形全是四边形,被称为四边形网格。}
\label{fig:tri_quad_cmp}
\vspace{-1mm}
\end{figure}

\begin{figure}[!h]
\centerline
{
\begin{overpic}[width=0.95\columnwidth]{introduction/tet_hex}
\put(22,0){\textbf{(a)}}
\put(71,0){\textbf{(b)}}
\end{overpic}
}
\vspace{1mm}
\caption{使用不同的多面体表示同样的三维体区域。图中展示的是三维体区域的切面。 (a)只用四面体组成了空间区域,称为四面体网格。 (b)网格上的多面体全是六面体,被称为六面体网格。}
\label{fig:tet_hex_cmp}
\vspace{-4mm}
\end{figure}

图~\ref{fig:mesh_graph_cmp}中给出了三个经常被使用的三维数字模型的例子,它们都是使用了多边形(多面体)来表示。图~\ref{fig:mesh_graph_cmp}(a)、(b)、(c)分别表示了只使用三角形单元表示一个二维区域,三角形单元表示一个三维曲面,四面体单元表示一个三维体区域。这样的离散三维数字模型被称为网格$\mM$。在本文中,研究的对象包括二维多边形网格,三维多边形曲面网格,三维多面体体网格。网格$\mM$包含了两个部分:拓扑部分和几何部分。网格$\mM$的拓扑部分可以使用一个无权无向\emph{图}来表示,\emph{图}中的顶点集合包含$N_v$个顶点$\{v_i, i=1,\ldots,N_v\}$。网格$\mM$ 的几何部分是通过每个顶点$v_i$上的三维位置$\mp_i$ 来表示:
\begin{equation}
\mp_i \triangleq \mp(v_i) = \left( x(v_i),y(v_i),z(v_i) \right)^T \in \mathbb{R}^3.
\end{equation}
使用网格$\mM$的几何信息可以用来计算网格其他的微分量,比如法向,曲率(高斯曲率,平均曲率)等,具体的计算方法可以见书\cite{Botsch2010}的第三章。\emph{图}的一种连接方式是通过$N_e$条边$\{\me_i, i=1,\ldots,N_e\}$将顶点相连。图的另外一种连接方式是使用$N$个网格单元(多边形或者多面体)$\{\ms_i, i=1,\ldots, N\}$ 来表示。这两种表示拓扑部分的方法是相互等价的。

根据使用的不同多边形或者多面体的种类不同可以对网格$\mM$ 进行分类。比如图~\ref{fig:tri_quad_cmp}中的(a)(b)表示了同一个三维曲面,但是(a)中的网格$\mM$只包含三角形,故称为三角形网格,(b)中的网格$\mM$只包含四边形,被称为四边形网格。图~\ref{fig:tet_hex_cmp}(c) 的四面体网格和图\ref{fig:tet_hex_cmp}(b)中的六面体网格都可以表达三维体区域。如果网格$\mM$中包含了少量的三角形,大量四边形,这样的网格通常称为四边形为主的网格。每种类型的网格各有自己的优缺点。在计算机辅助设计和有限元方法中,四边形、六面体网格会比三角形、四面体网格在一些问题中更加适合,它们能使用更少的单元获得更高的模拟精度;但是三角形、四面体网格比四边形、六面体网格更容易获得,生成,导致在其它很多的应用中它们更受欢迎。因此需要根据不同的应用,来决定使用什么类型的网格。%比如在有限元方法中,相比于四面体网格,六面体网格可以使用更少的网格单元,获得更高的模拟精度。

本文中我们研究的对象的是流形(manifold)网格。网格中没有非流形的边(面)和非流形的顶点\cite{Botsch2010}。比如,对于多边形网格而言,非流形的边会拥有大于2个相邻多边形;对于多面体网格而言,非流形的面会拥有超过2个相邻多面体;%$T$ 节点也是非流形的顶点。


\section{数字几何处理算法} \label{sec:DGP}
计算机图形学中的很多应用的重要一步是将原始的三维模型变换到其他的定义域中,或者变化成其他的形状。也就是在满足某些约束的前提下,计算一个映射$f: \Omega \subset \mathbb{R}^d \rightarrow \mathbb{R}^d$。映射计算是一个非常抽象的概念,但是在计算图形学中有很多的应用可以将之具体化。比如\emph{纹理映射},目标是把一幅图像贴到三维网格的表面上来增强真实感;\emph{网格变形},目标是对输入网格进行变形,成为另外一个形状,比如计算立正的人抬一下手后的形状;\emph{网格质量提高},目标是通过修改网格的顶点的位置,从而提高网格的质量;\emph{对立方体映射},目标是将输入的网格映射成一个多立方体结构,多立方体结构的简洁性使得在原网格上不能做的事可以被容易实现,比如用来生成高质量的六面体网格。具体来说,\emph{纹理映射}算法首先将三维网格一一映射到一个二维平面区域上(这个过程被称为\emph{网格参数化}),这个二维平面区域是和一个图像一一对应的,然后将这个图像逆映射回三维网格上,并且将图贴在网格上。图\ref{fig:texture_mapping} 展示了一个纹理映射的例子。正因为这个问题的重要性,在本文中,计算映射也是我们研究的重点,第\ref{chap:AMIPS}章和\ref{chap:affine}章都在研究了如何高效的计算一个高质量,低形变的、无翻转的映射,并且和之前的算法相比,把它们应用到\emph{纹理映射},\emph{网格变形},\emph{网格质量提高}上后产生的结果有显著的提高;第\ref{chap:polycube} 章讲述了如何利用法向的信息去构造一个映射,使得输入网格在这个映射下变成一个多立方体结构。

\begin{figure}[t]
\centerline
{
\begin{overpic}[width=0.95\columnwidth]{introduction/texture}
\put(20,-1){\textbf{(a)}}
\put(72,-1){\textbf{(b)}}
\end{overpic}
}
\vspace{1mm}
\caption{纹理映射。(a)将三维模型映射到二维平面上后,形成的二维三角形网格,颜色代表了映射形变的大小,越白越低。(b)将棋盘格图像投影回曲面网格后的贴图结果。}
\label{fig:texture_mapping}
\vspace{-5mm}
\end{figure}

通常现代三维获取设备获得的几何数据都是点云,也就是说只有几何部分(点的空间位置),没有拓扑部分(点与点之间的连接关系),于是在大部分情况下,处理的第一步就是重建曲面,建立点与点之间的连接关系。一般重建出来的网格$\mM$ 是三角形网格。但是如上文所述,不同的应用需要不同类型的离散化网格,所以对于重建的三角形网格$\mM$而言,有需求去生成不同类型的网格(比如四边形网格,四面体网格,六面体网格)或者符合其他要求的网格(比如各向同性网格,各向异性网格),这个生成的过程被称为\emph{网格生成}。但是直接重建出来的网格质量一般比较差,这时需要进行\emph{网格优化},以便它能够更好地被后面的应用使用。网格$\mM$的整体质量一般可以通过每个网格单元的质量的统计值进行刻画。通过测量每个网格单元和对应的正网格单元(正三角形,正四面体等)之间的“距离”可以度量它的质量。\emph{网格优化}的过程一般同时优化网格的几何部分和拓扑部分来提高质量,这个过程一般要求不改变网格单元的类型,比如三角形网格最终还是三角形网格,不会变成四边形网格或者其他类型的网格。因此\emph{网格优化}可以认为是\emph{网格生成}的一个特例,只是优化过程中网格单元的类型没有改变而已。
在本文中,我们将重点研究\emph{网格生成},第\ref{chap:LCT}章研究了如何在给定定义域和一般化的黎曼度量的前提下,生成高质量的各向异性网格,图\ref{fig:aniso_gene} 展示了一个各项异性网格生成前后的样子,这个过程也可以被理解为各向异性网格优化;第\ref{chap:polycube}章研究了如何生成低形变的多立方体结构,从而利用多立方体结构的特殊性去生成高质量的六面体网格。当网格生成完成后,可以利用最优映射计算来提高网格的质量,图\ref{fig:improve_tet}和图\ref{fig:improve_hex}展示了两个例子,分别提高各向异性四面体网格质量和六面体网格质量。

\begin{figure}[h]
\centerline
{
\begin{overpic}[width=0.95\columnwidth]{introduction/aniso_generation}
\put(20,-2){\textbf{(a)}}
\put(72,-2){\textbf{(b)}}
\end{overpic}
}
\vspace{2mm}
\caption{各项异性网格生成。(a)输入的网格,网格离指定的各项异性很远,质量很差。(b)生成的高质量各向异性网格。}
\label{fig:aniso_gene}
\vspace{-5mm}
\end{figure}

其他的数字几何处理算法一般还包括\emph{模型平滑},\emph{模型简化},\emph{模型修补},\emph{模型编码}等。这些不是本文的研究重点,在此不予赘述,如果大家感兴趣的话,详情可参见\cite{Botsch2010}。

\section{最优映射的计算}
在计算机图形学与数字几何处理中,有一项重要和基本的任务是,如何计算一个\emph{有效的映射}。比如网格参数化定义为:给定一个输入的二维流形三角网格$\mM$和一个二维流形参数域$\mathbb{R}^2$,寻求一个从参数域到$\mM$的一一映射,使得参数域上的网格与原始网格拓扑同构,并在保证参数域上三角形不重叠的同时,谋求某种与原始网格之间的几何度量的变形最小化。这种映射可以简单的理解为,在某些约束下,对输入模型的顶点计算一个新的位置,这是可以被利用到很多应用中的,比如平面网格参数化、网格变形、网格质量提高。

\subsection{最优映射的要求}
\emph{最优映射}$f: \Omega \subset \mathbb{R}^d \rightarrow \mathbb{R}^d$需要满足如下的要求。
\begin{itemize}
    \item $f$是无翻转的,也就是说$\det J(f) > 0$,其中$J(f)$是映射$f$的Jacobian矩阵;
    \item $f$的形变尽量小,也就是变换前后的某种几何度量尽量小,特别是能控制最大的形变,代表没有特别坏的映射出现;
    \item $f$是光滑的,在变形应用中非常需要;
    \item $f$的计算要足够高效,以至于能提供交互级的反馈速率。
\end{itemize}
\begin{figure}[t]
\centerline
{
\begin{overpic}[width=0.95\columnwidth]{iso_para_cmp}
\put(20,-2){\textbf{(a)}}
\put(69,-2){\textbf{(b)}}
\end{overpic}
}
\vspace{2mm}
\caption{Bunny模型的头部进行刚性参数化比较。左下角的区域是二维平面参数化结果,颜色代表了刚性形变的大小,越白越好;上面的区域是网格贴图后的结果,二维平面的图像是规则的方格图片。(a) 有效的映射:没有三角形翻转的参数化结果。(b) 不是有效的参数化,左下角图中的黄色的区域和上面图中绿色圈出来的区域是翻转的三角形,依照这样的映射使得原本正交的棋盘格纹理变得不相交了。}
\label{fig:iso_para_cmp}
 \vspace{-3mm}
\end{figure}

\begin{figure}[t]
\centerline
{
\begin{overpic}[width=0.95\columnwidth]{introduction/hex_opt}
\put(20,-2){\textbf{(a)}}
\put(69,-2){\textbf{(b)}}
\end{overpic}
}
\vspace{2mm}
\caption{Fertility六面体网格模型质量提高前后的比较。模型上的颜色代表了六面体的缩放Jacobian行列式的大小,颜色越白代表值越好,黄色代表值小于0(最坏的情况,不希望看到的)。 (a)质量提高前的六面体网格,存在很多缩放Jacobian行列式小于0 的单元; (b) 质量提高后的结果,平均与最差缩放Jacobian行列式都得到了大幅改善。}
\label{fig:hex_opt}
 \vspace{-3mm}
\end{figure}
如果网格单元上的$\det J(f) \leq 0$,我们称这个网格单元翻转。图~\ref{fig:iso_para_cmp}中比较了同一个模型上的两个刚性参数化的结果,不满足以上要求的参数化会导致\emph{纹理映射}的失败。网格参数化和变形是计算机图形学中的常见应用,这些应用的核心任务是寻找一个低形变的映射。正网格单元普遍被认为是质量最好的单元。如果从正网格单元到输入网格单元的映射的形变比较大时,说明这个网格单元离正网格单元比较远,也就是质量比较差,于是优化出一个低形变的映射(从正网格单元到当前网格单元),可以提高网格的质量,这时最大的形变会更加重要,因为一个比较差的单元可能会导致有限元方法等应用的失败,图\ref{fig:hex_opt}展示了一个六面体网格质量提高前后的结果对比。

\subsection{相关工作}
如何生成高质量的映射,在最近吸引了很多的关注。很多的新技术被提出来,我们在这给出一些相关的工作。

\paragraph{基于网格的映射} 在过去的二十年,基于网格的映射的很多技术被开发出来(见综述~\cite{Floater2005,Botsch2010})。为了高效的计算映射,很多算法都是基于线性的(目标是二次能量函数),这些方法不能保证结果是无翻转的,也就是在某些网格单元上的$\det J(f) \leq 0$。 比如,网格参数化的尽量共形的方法(Least-Square Conformal Mapping,LSCM)~\cite{Levy2002} 能够获得很低的平均共形形变,但是如果网格的切缝太短,LSCM可能会产生翻转的三角形。平均值坐标~\cite{Floater2003mean} 方法也有可能引入高形变,甚至翻转的网格单元。尽可能刚性的(As-Rigid-As-Possible,ARAP)方法~\cite{Liu2008,Chao2010}同样不能防止翻转。基于角度展平的网格参数化\cite{Sheffer2001,Sheffer2005,Zayer2007}能获得很小的共形形变,但是不能保证参数化结果中无翻转的网格单元。文献\cite{Sheffer2005} 尝试在最后一步通过移动顶点的位置将翻转的单元翻回来,但是在我们的实验中(原文作者的实现),它的结果中依然存在翻转的网格单元。
虽然这些方法不能保证生成无翻转的映射,但是它们的结果可以作为我们算法(第\ref{chap:affine}章)的输入。%我们能够使用现有的方法来高效的获得初始映射。

\paragraph{网格质量提高}
最优网格生成是数字几何处理中一个很重要的问题,可以用来生成高质量的网格(比如\cite{bern1992mesh,owen1998survey,teng2000unstructured,eppstein2001global,shewchuk2011unstructured})。这些算法虽然可以生成整体上具有不错的质量网格单元,但是依然有提高的空间。对于二维三角形网格而言,常用的方法有通过在质量比较差的网格单元上插入Steiner点\cite{ruppert1995delaunay},通过移动点的位置来提高单元质量\cite{amenta1999optimal}。对于四面体网格而言,比如\cite{Klingner2007}设计了一个网格质量提高操作集,实现程序Stellar用来实际地提高质量;比如\cite{Freitag2002}优化四面体上的Frobenius条件数,同时利用一些启发式的方法避免目标函数的不光滑性。对于四边形网格而言\cite{bommes2013quad},一类方法通过切向光滑来优化\cite{remacle2012blossom,tarini2010practical},另外的一些方法会修改网格连接关系来优化\cite{bommes2011global,tarini2011simple}。 对于六面体网格而言,\cite{Frey2008,zhang2009surface}通过移动点的位置来优化,但是它们经常在网格的凹区域中引入翻转的六面体。 \cite{Livesu:2015:Untangler}提出了使用圆锥体形状优化(一个全局的优化方法)去全局更新顶点,虽然没有理论保证能生成没有翻转的六面体网格,实际中,它还是能提供很好的结果。我们是通过局部移动顶点的方式进行质量提高的,图1.12展示了一个六面体网格质量提高前后的结果,我们的算法能产生很高质量的六面体网格。

\paragraph{无翻转的映射} 计算无翻转映射问题可以描述成一个非线性带约束优化问题,它的目标函数刻画了映射的形变,约束描述了无翻转的要求。最近很多技术被提出用来求解这类问题。其中一大类方法需要一个无翻转的初始化,然后在优化的过程中一直维持在无翻转的解空间中,并且设计的目标函数一般有如下的性质:当$\det\, \mJ(f)$ 趋向于0时,目标函数趋向于无穷大。这样可以将一个有约束的非线性优化问题转化为无约束优化问题。
实际中,无翻转的初始化是可以获得的,比如对于网格参数化,初始的映射能够通过Tutte嵌入算法~\cite{Tutte1963,Floater2003}获得;对于网格变形,变形前的网格可以作为初始的无翻转的网格。
研究人员设计了很多满足上述性质的目标函数,比如形变最小化的参数化(Most-Isometric ParameterizationS,MIPS)能量~\cite{Hormann2000}和它的变种~\cite{Degener2003,Fu2015,Smith2015},或者其他拥有障碍函数的能量~\cite{Schuller2013,Jin2014}(障碍函数用来阻止退化的网格单元和抑制高形变)。
但是这类方法还存在一些缺点。比如在网格变形中,如果只存在少量控制点约束,这类方法通常能获得无翻转的映射,同时拥有很低的形变。但是如果存在很多的控制点约束,比如计算固定边界的映射,这类算法会因为控制点之间的相互制约,而导致收敛很慢,甚至失败。 为了解决这些问题,通过观察到网格的三角化会影响映射和计算的收敛速度,Jin 等~\cite{Jin2014} 对于二维网格变形,提出了自适应的重新网格化来扩展解空间;Weber等~\cite{Weber2014} 对于二维固定边界的映射,允许在原始和目标网格上插入新的顶点。但是它们只能在二维网格上工作(网格参数化的输入是一个三维曲面网格,但是在每个三角形上建立局部坐标系后,映射就是二维的,如图\ref{fig:para_example} 所示)。 当位置的约束被强加时,为了进一步的提高计算映射的效率,~\cite{Schuller2013} 提出了\emph{子步长}策略,使用中间点位置来代替目标控制点位置。子步长策略创造了控制点的移动轨迹,用来辅助优化。然而,对于固定边界的映射,存在很多的位置约束,该方法很难成功,因为控制点的移动轨迹存在相互冲突~\cite{Jin2014,Fu2015}。

\paragraph{形变有界的映射} Lipman和他的合作者提出了新颖的算法来显式地对映射的形变的上下界做约束。 他们的方法\cite{Lipman2012,Aigerman2013,Kovalsky2014,Poranne2014,Chen2015}寻找形变有界的映射,而且不需要无翻转的初始化,可以推广到$n$ 维空间上的映射。他们通过一个最大凸子空间近似化受约束的空间,并且求解一个受约束的非线性优化问题。当他们的算法收敛时,形变有界的无翻转的映射可以被求出。然而,他们方法的计算速度和尺度扩展性是主要的瓶颈。因为在他们的受约束的非线性问题中,大量使用了代价颇高的二次规划,半正定规划。最近Kovalsky等~\cite{Kovalsky2015}提出将约束线性化和矩阵预分解技术加速形变有界算法。虽然他们的算法足够高效,但是设置一个较小的形变的界限值可能会导致算法失败。实际应用中,用户无法预先得知一个合理的形变界限应该是多少,所以,他们的方法无法保证成功的收敛,只能通过用户不停的尝试才能确定一个比较合理的界限。因此他们的算法依然没有任何的理论,来保证成功收敛。我们的方法(第\ref{chap:affine}章)也能够获得形变有界的映射,并且可以被利用来快速寻找最优的形变界限。Levi和Zorin~\cite{Levi2014} 在最优化过程中,通过锥优化去优化ARAP 能量的$L_\infty$范数,尝试去控制最大的形变,但是不能保证无翻转的网格单元。

\paragraph{形变度量} 标准的二维MIPS能量测量了映射的共形形变:$\sigma_1\sigma_2^{-1} + \sigma_2  \sigma_1^{-1}$,其中$\sigma_1,\sigma_2$是映射的Jacobian矩阵的奇异值。为了抑制刚性形变,Degener等~\cite{Degener2003} 设计了形如$(\sigma_1\sigma_2^{-1} + \sigma_2  \sigma_1^{-1})(\sigma_1\sigma_2+\sigma_1^{-1}\sigma_2^{-1})^\theta$的能量函数,这一能量函数可以同时抑制了共形形变和体积形变,从而抑制刚性形变。其他的能量,比如Dirichlet 能量$\sigma_1^2+\sigma_2^2$,拉伸能量$\max\{\sigma_1,\sigma_2\}$,Green-Lagrange能量$(\sigma_1^2-1)^2+(\sigma_2^2-1)^2$,也是使用Jacobian矩阵的奇异值来设计的。尽可能刚性(ARAP),尽可能相似(ASAP)和AKAP也是三种不同的方法来最小化刚性/共形扭曲~\cite{Alexa2000,Igarashi2005,Sorkine2007,Solomon2011},但是他们的优化算法~\cite{Liu2008,Solomon2011}不能保证映射是无翻转的。这些方法使用平方和将形变累加,不能抑制最大的形变或者控制形变的分布。Weber等~\cite{Weber2012}计算极限拟共形变换,在满足边界条件的情况下,可以得最小化最大的共形形变。

\paragraph{块坐标轮换下降算法(BCD) }也被称为\emph{非线性块高斯-赛德尔方法},是一种非常流行的优化算法,可以用来求解大规模的问题,特别是结构化的问题。基本的想法是固定其他的变量,一块一块的更新某些变量。标准的BCD方法在最小化子问题时,需要子问题达到收敛,然后转换到其他的块变量进行优化。这也被称为\emph{精确块坐标轮换下降算法}(exact BCD)。它已经被广泛的应用到计算机图形的应用,比如MIPS参数化~\cite{Hormann2000},网格质量提高~\cite{Diachin2006}。 但是exact BCD 的缺点是他的收敛速度比较慢,在寻找子问题最优解的过程需要的计算量比较高。最近,\emph{非精确块坐标轮换下降算法}(inexact BCD)被提出来克服这些问题~\cite{Bonettini2011,Tappenden2013,Cassioli2013}。Xu和Yin~\cite{Xu2013}展示了在某些条件下,inexact BCD 能够获得更低的目标函数,并且不容易陷入局部最小。在第\ref{chap:AMIPS}章中,我们展示了inexact BCD非常适合计算低形变的无翻转映射。

\paragraph{变换的表达} 在形状变换中,将变量表达为网格顶点位置是最常用的做法。但是,对于某些应用,存在其它更合适的表达方式,可以使得计算效率和鲁棒性获得显著提高。基于角度的网格表达非常适合于二维共形参数化~\cite{Sheffer2001,Sheffer2005},它的三维版本 -- 基于二面角的表达 -- 被证明在四面体映射中非常有用~\cite{Gilles2015}。度量缩放,网格边长的缩放,对于共形嵌入~\cite{Springborn2008,Ben-Chen2008}是另外一个很好的表达。Crane等~\cite{Crane2011,Crane2013} 操作曲率空间,来控制旋转变换和Willmore曲率流。我们在第\ref{chap:affine}章中,使用分段线性的仿射变换作为变量来直接控制形变和避免翻转。

\section{各项异性网格生成}
在三维网格被使用前,我们需要对三维网格模型进行质量优化或者生成满足特定要求的网格,其中重新网格化是一种常见与重要的技术。重新网格化被定义为\cite{Botsch2010,Alliez2008}:
\begin{definition} \label{def:remesh}
给定一个输入网格,计算另外一个输出网格,使得输出网格的单元满足某些质量准则,并且在可接受范围内近似输入网格。
\end{definition}
定义\ref{def:remesh}中的“近似”能够理解为输出网格的位置,法向,或者更高阶的微分性质和输入网格是接近的,实际中一般使用输入输出网格之间的Hausdorff距离进行度量。不同的应用会需要不同的质量准则和不同类型的网格。比如在几何建模、物理模拟、机械工程中,与曲面曲率一致的或者和偏微分方程解的Hessian矩阵一致的各向异性网格有很大的需求。在对原始模型逼近精度类似的前提下,它能够使用比各向同性网格更少的网格单元,同时可以在数值模拟中获得更高的精度。因此我们需要对输入的网格进行重新网格化,得到满足要求的高质量各向异性网格,如图\ref{fig:aniso_gene}所示,这个过程被称为\emph{各向异性网格生成}。

\subsection{各项异性的数学描述}
\begin{figure}[t]
\centerline
{
\begin{overpic}[width=0.95\columnwidth]{introduction/iso_aniso}
\put(20,-4){\textbf{(a)}}
\put(69,-4){\textbf{(b)}}
\end{overpic}
}
\vspace{4mm}
\caption{同一个模型上的各向同性(a)与各向异性(b)网格比较。各向同性网格上的三角形都接近于正三角形(a),各向异性网格上的三角形的大小,朝向,长宽比和曲面曲率是一致的(b)。}
\label{fig:iso_aniso_cmp}
\vspace{-1mm}
\end{figure}

\begin{figure}[t]
\centerline
{
\begin{overpic}[width=0.95\columnwidth]{introduction/aniso_input}
\put(12,-3){\textbf{(a)}}
\put(46,-3){\textbf{(b)}}
\put(82,-3){\textbf{(c)}}
\end{overpic}
}
\vspace{4mm}
\caption{三种不同的各向异性输入。\emph{逆变换}将单位圆(球)映射成椭圆(球),椭圆(球)的长短轴的方向,大小,比例是和目标各向异性的要求一致的,因此图中的椭圆(球)可以用来表示黎曼度量。(a) 定义域是一个二维方形区域,各向异性是由一个全局凸函数的Hessian矩阵诱导出。(b) 定义域是一个三维曲面网格,各向异性是通过曲面的曲率定义的。(c)定义域是一个三维立方体区域,各向异性是一个给定的,设计好的黎曼度量场$\mathcal{M}$。}
\label{fig:aniso_input_cmp}
\vspace{-3mm}
\end{figure}
\emph{各向异性}描述的是在不同的空间位置,网格单元(三角形或四面体)具有不同的\emph{朝向},\emph{大小},\emph{长宽比}。图~\ref{fig:iso_aniso_cmp} 展示了同一曲面的各向异性网格与各向同性网格之间的比较。生成各向异性网格的输入是一个黎曼度量场$\mathcal{M}$ 和一个定义域$\Omega \subset \mathbb{R}^d$,输出是定义域上满足输入黎曼度量的网格。图\ref{fig:aniso_input_cmp}展示了三种不同的输入定义域和黎曼度量场。我们希望设计的算法可以适用于一般化的黎曼度量场$\mathcal{M}$,比如通过凸函数的Hessian矩阵诱导出的黎曼度量场,直接设计的解析黎曼度量场,或者使用曲面的曲率信息设计的黎曼度量场。在每个顶点$\mp \in \Omega$上的各向异性可以被表达为一个$d\times d$ 的对称正定矩阵$M(\mp)$,称为\emph{增广度量}。它的特征值分解为$M(\mp) = Q \, \Lambda \, Q^T$,这个分解对目标各项异性有很好的描述:
\begin{itemize}
    \item \emph{朝向}: $Q$ 的列向量指定了网格单元的目标朝向;
    \item \emph{大小}: 对角矩阵$\Lambda$中的每个非零单元指定了网格单元在每个目标朝向上的大小;
    \item \emph{长宽比}: \emph{各向异性比例}被定义为:
\begin{equation}
\lambda \triangleq \sqrt{\max \lambda_i / \min \lambda_i}.
\end{equation}
它表达了网格单元在目标朝向上的长宽比,其中$\lambda_i$是$M$的特征值($\Lambda$ 的对角单元)。
\end{itemize}

如果输入的$\mathcal{M}$中存在很大的各向异性比例$\lambda$,代表了这个场的变化比较剧烈,这会增加各向异性网格生成的难度。我们希望本文的算法可以处理变化剧烈的黎曼度量场。在增广度量下,定义域中的任意两点$\mp, \mq \in \Omega$ 之间的线段$e$ 的\emph{各向异性长度}定义为:
\begin{equation}\label{equ:aniso_distance}
|e_{\mathcal{M}^{-1}}|(\mp, \mq) \triangleq \int_{0}^1 \sqrt{(\mp-\mq)^T \, M\left(t\mp+(1-t)\mq\right)\, (\mp-\mq)} \,\mathrm{d}t.
\end{equation}
当$\mM$变化比较慢和$\mp, \mq$比较近时,它可以被近似为:
\begin{equation} \label{equ:app_aniso_distance}
|e_{\mathcal{M}^{-1}}|(\mp, \mq) \approx \sqrt{(\mp-\mq)^T \, (M_\mp+M_\mq)/2 \, (\mp-\mq)}.
\end{equation}
各项异性网格生成的目标$g_1$:输出网格的边的\emph{各向异性长度}是一样长的。在欧式空间中,通过$\Lambda^{-1/2} Q^T$可以将一个各向异性网格单元变换到一个各向同性网格单元,这个变换被称为\emph{逆变换}。各项异性网格生成的另外一个目标$g_2$:输出网格的网格单元在\emph{逆变换}下,都是边长一样长的正网格单元(正三角形,正四面体)。显而易见,$g_1$和$g_2$是相互等价的。从生成的网格离这两个目标的远近程度来度量算法好坏。

\subsection{相关工作}
在过去的20年中,很多各向异性网格生成技术被提出来了。我们在这给出了一些相关的工作。

\paragraph{基于Delaunay三角化的方法}
通过扩展Delaunay三角化的外接圆性质可以将Delaunay三角化一般化并用来生成各向异性网格~\cite{Thompson1998,Frey2008}。在各项同性网格生成中,可以在质量比较差的单元中插入根据Delaunay核定义的Steiner点来改善网格的质量。于是可以基于各向异性Delaunay核,修改Steiner点的定义,并且将之插入到质量差的网格单元中,将这个方法扩展到各向异性网格生成中~\cite{Frey2008,Dobrzynski2008}。其他的方法通过局部操作来细化三角化~\cite{Hecht1998,Jiao2010},局部操作是根据网格质量确定的。Boissonnat等提出了一个变种算法:局部一致的各向异性网格生成。当黎曼度量场和定义域的边界是平滑的时候,通过顶点的插入操作可以在理论上保证网格的质量~\cite{Boissonnat2008,Boissonnat2011,Boissonnat2013}。这类方法虽然能够保证最终的网格的质量,但是一般只有一种质量可以被保证,所以最终网格的整体质量比较差。而且这类算法的效率都比较低,而且只能处理变化缓慢的黎曼度量场(各向异性比例$\lambda$在定义域中变化比较缓慢),导致算法不够实用。

\paragraph{各向异性Voronoi方法}
Voronoi图可以使用各向异性测地线距离进行扩展,获得\emph{各向异性Voronoi图}(Anisotropic Voronoi Diagram,AVD)。点集 $\{\mp_i \in \mathbb{R}^d \}_{i=1}^n$ 将欧式空间$\mathbb{R}^d$ 剖分为一个不连接的多面体集合$\{\Omega_i\}_{i=1}^n$,其中$\Omega_i = \{\mx \in \mathbb{R}^d \: | \: \dis(\mx, \mp_i) \leq \dis(\mx, \mp_j) \, , \forall \, j \neq i\}$,$\dis$是距离函数。如果$\dis(\mx, \my) = \|\mx-\my\|^2$,AVD退化到一般的Voronoi图。当在$(\Omega, \mathcal{M})$上测量黎曼距离$\dis(\mx, \my) = |e_{\mathcal{M}^{-1}}|(\mx, \my)$(见公式\ref{equ:aniso_distance})时,剖分变为\emph{各向异性Voronoi 图},$e$是$\mx$与$\my$之间的边。AVD的对偶,\emph{各向异性Delaunay三角化}(Anisotropic Delaunay Triangulation,ADT),在某些情况下,才是一个单纯形网格(流形),不能够保证输出是一个正确的三角形网格。通过顶点的插入~\cite{Labelle2003,Cheng2006}或者基于各向异性中心Voronoi 细分(Anisotropic Centroidal Voronoi Triangulations,ACVT)能量函数优化顶点位置~\cite{Du2005,Valette2008,Levy2010,Levy2012},可以优化一个初始的AVD,从而使AVD的对偶ADT和期望的各向异性一致。然而优化AVD的方法目前只在只能在二维上或流形曲面成功,而且由于优化过程中每次迭代中的构造AVD 和优化ACVT 能量效率都很低。过多的迭代次数导致整个算法的效率非常低下。而且AVD的对偶ADT 不能够保证是一个流形网格~\cite{Canas2011}。

\paragraph{基于粒子的方法}
通过粒子之间的斥力或者引力对顶点进行位置优化~\cite{Shimada2000,Persson2004}在网格生成中是一个流行的算法。基于粒子的算法~\cite{Shimada2000,Zhong2013} 利用斥力来建立粒子之间的平衡,计算这些粒子的AVD,最后取对偶得到ADT,使最终的三角形网格到达目标各向异性的要求。$\exp\left(-\frac{|e_{\mathcal{M}^{-1}}|^2(\mp_i,\mp_j)}{4\sigma^2}\right)$表达了两个粒子$\mp_i$和$\mp_j$之间的能量\cite{Zhong2013},它的导数的相反数表示了两个粒子之间的力。将所有两两粒子之间的能量相加进行优化,从而更新粒子的位置,并且将粒子投影回原始定义域上(比如顶点不能离开输入三维曲面定义域),直至能量到达最小(力达到平衡)。最终的网格可以通过计算粒子的AVD的对偶ADT获得。基于粒子的方法的最主要的优点是它很简单,与网格无关。但是当各向异性变化剧烈时,核函数的宽度$\sigma$和其他参数的选择会对最终生成的网格质量产生很大的影响。和各向异性Voronoi方法一样,由于ADT不能保证生成流形,所以需要后处理来保证输出网格是一个流形。

上面三类方法都存在两个局限性。第一,他们的计算复杂度高。在复杂的定义域中,构造与优化AVD,从相邻的粒子聚集斥力,计算黎曼测地距离都是非常费时的。第二,这类算法的结果质量比较差,所以在生成符合目标各向异性的网格这个问题上还留下了很大的余地。其他的一些方法~\cite{Labelle2003,Boissonnat2008,Boissonnat2013}尝试解决这个问题,以便能够将某些质量度量约束在界内,但是这些方法只适用于变化比较缓慢的各向异性黎曼度量场$\mathcal{M}$,并且不适用于拥有尖锐特征的模型。因为这些方法只能够约束住某些质量度量,不是所有的质量度量,所以从视觉上或者质量度量的统计上来看,整体网格的质量依旧不是很好,也不能很好地符合目标各向异性黎曼度量场$\mathcal{M}$(见~\ref{sec:LCT_cmp}节的比较)。

\paragraph{基于函数插值的方法}
各向异性网格的质量能够通过偏微分方程的连续解$u$与它在网格上的分段线性插值函数$\hat{u}$之间的误差来衡量~\cite{Zienkiewicz2005}。Chen等提出了\emph{最优Delaunay三角化}(ODT)来优化误差函数$\|u-\hat{u}\|_{L^p(\Omega)}$~\cite{Chen2004,Chen2007}(输入连续函数与离散插值函数之间的$L^p$误差)。当$p=1$和$u(\mx)=\mx^2$ 时,ODT被成功地推广到各向同性三角形网格\cite{Chen2004,Chen2007,chen2012isotropic}与四面体网格生成中~\cite{Alliez2005,Tournois2009,Chen2011}。 当黎曼度量场$\mathcal{M}$是通过一个全局凸函数的Hessian矩阵诱导出来时,ODT能够被自然的推广到各向异性网格生成中,并且三角化变成正则三角化(或者被称为加权Delaunay 三角化)。这个性质被很多应用利用,包括自支撑曲面设计~\cite{Liu2013},电阻断层扫描~\cite{Desbrun2013}和其他的几何处理应用~\cite{Mullen2011,Goes2014}。 但是一般情况下,一个一般的黎曼度量场$\mathcal{M}$ 不是任何全局凸函数的Hessian矩阵~\cite{Boissonnat2008a,Amari2014},也就是说不存在一个全局凸函数的Hessian矩阵与输入的各向异性黎曼度量场$\mathcal{M}$一致。\cite{Chen2004a}提出了,在一个顶点的一领域中\emph{局部}的使用ODT,但是该方法失败了,如图~\ref{fig:exp}所示。我们的方法可以克服这些问题,适用于一般化的黎曼度量场$\mathcal{M}$,能高效的生成高质量的结果。

\paragraph{各向同性网格生成}
我们在这里只讨论常见的三角形(综述\cite{Alliez2008,Botsch2010})和四面体网格的生成。ODT~\cite{Chen2004,Chen2007,chen2012isotropic}和CVT\cite{liu2009centroidal,yan2009isotropic}是两个主要的算法来生成各向同性三角形网格。\cite{botsch2004remeshing} 提出了一种非常高效的组合算法,使用的基本组件包括劈边,合并边和顶点平滑等操作。我们生成各向异性网格的算法受到了\cite{botsch2004remeshing}的启发,通过劈边与合并边操作可以使初始网格快速的和目标接近吻合。
~\cite{Alliez2005,Tournois2009,Chen2011}推广了ODT,用于生成各向同性四面体网格。Tetgen\cite{Si2015}是一个很鲁棒的程序,可以用来生成受约束的Delaunay四面体网格。我们在生成各向异性网格和多立方体结构的时候,都是使用TetGen 来生成初始四面体网格的。

\section{多立方体结构生成}
\begin{figure}[t]
\centerline
{
\begin{overpic}[width=0.95\columnwidth]{introduction/tet_pc_hex}
\put(16,-2){\textbf{(a)}}
\put(50,-2){\textbf{(b)}}
\put(84,-2){\textbf{(c)}}
\end{overpic}
}
\vspace{2mm}
\caption{将输入四面体Kiss模型(a)映射成多立方体结构(b),利用(b)的结构生成高质量的六面体网格(c)。(c)中的颜色代表了六面体的缩放Jacobian行列式的大小,越白代表值越好。}
\label{fig:tet_pc_hex}
 \vspace{-3mm}
\end{figure}

从三维体网格中抽象出的多立方体结构在计算机图形学应用中是十分有用的,比如纹理映射~\cite{Tarini2004,Yao2008,Chang2010},六面体网格生成~\cite{Gregson2011},基于GPU的细分~\cite{Xia2011}。多立方体结构的基本要求:最终网格的每个表面三角形的法向都是和坐标轴X,Y,Z对齐的,也就是法向只能取6种可能($(\pm 1,0,0)^T$, $(0,\pm 1,0)^T$ 或者 $(0,0,\pm 1)^T$),并且表面三角形的三个顶点的位置中一个分量是一样的。比如三角形的法向是和$(0,\pm 1,0)^T$对齐的,那么这个三角形上的顶点的Y分量是一样的。
高质量的多立方体结构在这些应用中是基本的要求。图\ref{fig:tet_pc_hex}展示了一个通过多立方体结构生成高质量六面体网格的例子。
\subsection{多立方体构造算法的要求}
一个高质量的多立方体构造算法应该满足下面的要求:
\begin{itemize}
    \item 算法是全自动的,没有手动的干预;
    \item 最终的多立方结构无翻转或者退化的网格单元,并且从多立方体映射的形变越小越好;
    \item 多立方体的奇异性是可控的,也就是说在算法中存在一个可调的变量,用来控制结果中奇异点的数目;
    \item 算法应该比较高效,否则手动的构造更受欢迎。
\end{itemize}

全自动且高效率是算法的基本的要求,也是我们算法的基本出发点。无翻转的要求在之前的文中已经多次提到,这也是很多后续应用的基础。奇异性和形变是构造多立方体结构的一对折衷量:如果奇异性很复杂,那么多立方体映射的形变会比较小;如果奇异性比较简单,则目标多立方体结构离输入网格比较远,导致映射的形变变大。折衷的最优解没办法数值化,因此我们希望能够找到一个参数,不同的值会导致最终的结果具有不同奇异性,可以供用户自行选择。

针对上面的目标,很多的算法已经被开发出来。一个经典的算法是基于网格分割与变形的。输入曲面网格被分割成不同的块,每块对应了不同的目标方向(上述六种可能方法之一)。通过变形将网格的法向和预先发现的方法进行匹配,以至于多立方体结构能够被获得。Gregson等~\cite{Gregson2011} 实现了这个想法,并且将它应用到六面体网格生成上。Livesu等~\cite{Livesu2013}使用基于图割(Graph-Cut)的算法提高了~\cite{Gregson2011}的标记算法的准确率。这些方法能够生成低形变的多立方体结构,但是他们对输入网格的朝向比较敏感,并且他们尝试求解非单调问题(或者被称为\emph{极限点},或者\emph{拐点})的方法是基于局部搜索,它是启发式的、耗时的。Huang等~\cite{Huang2014}提出了基于法向$l_1$ 模的算法,能够同时找到最优的朝向和低形变的映射。但是由于复杂的优化算法,他们的算法比较费时。我们的算法也是基于网格变形的,并且希望满足上述的所有要求。

\subsection{相关工作}
我们在这给出了自动构造多立方体结构相关的工作。

\paragraph{自动多立方体构造}
手动构造多立方体结构~\cite{Tarini2004}是非常费时的,特别是对于复杂的物体。基于方格或者八叉树分解的方法能够很自然地获得多立方体结构,但是结果会过于的稠密,对应的多立方体映射的形变非常大。Lin等~\cite{Lin2008}通过网格分割和方盒拟合构造多立方体结构。他们方法的最大问题是结果的多立方体结构过于简单(奇异点的数目过少),以至于产生很大的形变。He等~\cite{He2009}通过在关键点上的扫描线策略渐渐的构造多立方体结构,其中关键点是通过调和函数决定的。他们的结果在简单的物体上会产生过于复杂的多立方体结构(奇异点的数目过多),而且体映射是分开计算没有考虑多立方体的生成带来的形变。Gregson等~\cite{Gregson2011}利用网格变形计算将输入体网格变形到一个多立方体结构上。他们的变形包括如下两步: 旋转驱动的变形将大部分的表面法向和它的最主要的轴对齐;位置驱动的变形将分割好的块压平,从而获得最终的多立方体结果。尽管他们的方法通常能生成一个低形变的映射,但是它不能对网格的奇异性(也就是奇异点的数量)进行控制,而且他们处理极限点的方法是启发式的,没有保证消除极限点。后续的方法~\cite{Livesu2013}通过基于网络图分割的方法克服了这些问题,但是因为局部贪婪的搜索算法使整体算法比较费时。最近Sokolov和Ray~\cite{Sokolov2015}通过解决法向对齐的约束来尝试解决非单调的问题。他们的方法可行,但是方法的鲁棒性没有得到很好的实验证明。Huang等~\cite{Huang2014}通过优化曲面法向的$l_1$范数去除掉输入网格的朝向的约束,并且可以提供奇异性控制。但是他们的非线性求解器比较费时,特别是在大尺度的模型上。Yu等~\cite{Yu2014}将~\cite{Gregson2011}的基于旋转的结果进行体素化,利用一系列的形态学的操作优化多立方体结果。他们能够去除极限点,并且从体素化的结果中直接和高效的找到多立方体结构。但是他们的结果和~\cite{Livesu2013,Huang2014}相比,依然过于稠密(奇异点的数目过多),并且体映射也是分开计算的。

\paragraph{多立方体映射计算}
在很多的应用中,找到一个从输入网格到多立方体结构的一一映射是很重要的,比如六面体化~\cite{HanXiaHe2010,Gregson2011}。很多的技术被设计用来寻找这样的映射。一些方法,比如平均值坐标~\cite{Floater2003mean},调和函数方法~\cite{LiGuoWangEtAl2007}比较流行,但是不能保证映射的双射性。最近的计算无翻转的映射算法~\cite{Schuller2013,Fu2015}通过最小化变形形变,保证无翻转来寻找体映射。但是对于固定边界的映射,比如多立方体映射,因为在边界上有太多的约束,他们经常失败。前面提到的形变有界的算法能够用来控制扭曲,但是因为大量使用了二次规划和半正定规划,导致计算复杂度比较高。Kovalsky等~\cite{Kovalsky2015}提出了一个一阶算法来加速。上面的这些方法都假设多立方体是给定的,以至于构造的形变没有被考虑。基于变形的方法~\cite{Gregson2011,Huang2014}在计算的过程中,输入四面体被变形到多立方体结构,以至于能够同时找到多立方体结构和映射。但是他们的算法可能产生翻转或者退化的四面体,因为在优化过程中没有变形能量去惩罚这种情况的出现。后处理需要用来处理这个问题。而我们可以使用AMIPS函数来获得低形变的映射。

\paragraph{六面体网格生成}
六面体网格在有限元方法等应用有很大的需求,但是在实际工作中,为了生成一个高质量和边界满足输入的六面体网格,即使熟练的用户需要使用几个小时,甚至几天。在过去的几十年中,六面体网格自动生成算法一直是网格生成的热点(见综述\cite{owen1998survey,shimada2006current,shepherd2008hexahedral})。在工业界中,经常使用的算法,如基于多扫略面的方法\cite{shepherd2000methods}和铺筑与修饰方法\cite{staten2005unconstrained},依然是半自动的,需要用户将原始模型分解成合适的子块。\cite{sheffer1999hexahedral}通过Voronoi图可以将体模型自动地分解,但是它对边界比较敏感。\cite{carbonera2006constructive}通过使用Geode-Template提出了一种构造性的算法,用来生成受约束的六面体化。实际上,\cite{marechal2009}提出了一种新颖的基于八叉树的算法,可以自动地减少六面体的形变并保持输入网格的尖锐特征。最近的一些方法用三维场来生成六面体网格\cite{Nieser2011,Li2012},但是这类方法在最后参数化的过程会生成翻转的六面体,而导致整个算法的失败。另外的一些方法就是利用多立方体结构~\cite{HanXiaHe2010,Gregson2011,Huang2014}来生成六面体网格。我们的算法也是基于多立方体结构的,而且能处理输入模型的朝向问题并同时很好地控制映射形变,可以生成高质量的六面体网格,具体的比较见第\ref{chap:polycube}章中的第\ref{sec:pc_results}节。


\section{本文的贡献及组织}
在\textbf{第\ref{chap:AMIPS}章}中,我们尝试去获得低形变的和计算高效的无翻转映射。我们提出了一个简单而且高效的非线性方法在二维和三维空间中,去寻找\emph{有效的映射}。我们的算法是基于著名的MIPS方法~\cite{Hormann2000}。 成功的关键点是:
\begin{itemize}
    \item 修改原始MIPS能量,对于基于网格的映射或者无网格的映射,能够显著地抑制最大的刚性/共形形变。
    \item 将MIPS方法的顶点优化策略更换,使用非精确块坐标轮换下降算法能获得更高效的优化算法,对于基于网格的应用,可以有效的避免优化算法过早地进入局部最小。
\end{itemize}
我们称我们的方法为AMIPS。AMIPS自然地继承了MIPS的无翻转性质。相比于其他的计算无翻转映射的算法~\cite{Lipman2012,Schuller2013,Levi2014},AMIPS能够获得相当或者更低的形变,并且有1至2个数量级的加速。我们成功地将我们的算法应用到不同的问题上,比如平面网格参数化,基于控制点的二维三角形,三维四面体网格变形,基于控制点的二维,三维无网格变形,各向异性四面体网格和六面体网格质量提高。对于中等数量的网格变形中,我们能够获得交互级的反馈。

在\textbf{第\ref{chap:affine}章}中,我们提出了一个计算低形变的无翻转映射的新颖方法。关键的想法是将输入网格分离成不相连的网格元素,每个网格单元都是不翻转的,在优化的过程中,每个单元映射的质量一直保持在可行解空间中。我们通过将分段恒定的仿射变换作为变量,把带约束的非线性问题转化为一个无约束问题。我们优化的目标是寻找低形变的分段恒定的仿射变换,并在某些约束下,将分离的网格单元组装起来。我们的贡献可以归纳为:
\begin{itemize}
  \item 我们提出了一个新颖的\emph{网格单元组装}的方法,用来计算无翻转映射。我们利用AMIPS来设计惩罚函数,用来对形变加界限并防止网格单元翻转。
  \item 我们的方法对初始映射非常鲁棒,可以处理初始有大量的翻转单元和众多的位置约束。
  \item 我们通过不同的应用:网格参数化,二维/三维固定边界的映射,网格变形,证明了我们的方法的高效性和鲁棒性。
\end{itemize}

在\textbf{第\ref{chap:LCT}章}中,我们扩展了基于函数逼近的方法,对于二维平面,三维曲面和三维体区域,提出了一种新颖的各向异性网格生成方法。关键的想法是:根据每个网格单元(三角形或者四面体)上面的黎曼度量,建立一个\emph{局部}的凸函数,对于每个单元建立类似于最优Delaunay 三角化(ODT)的插值能量,将所有单元上的能量相加得到最后的目标函数。通过优化这个能量,更新顶点的位置和网格的链接关系,得到最后的高质量网格。同时我们的算法中还会加入一些局部的几何操作来满足一些几何准则。我们的贡献可以归纳如下。
\begin{itemize}
	\item 我们提出了\emph{局部凸函数三角化} (LCT),这是一种推广的最优Delaunay三角化(ODT)的方法,能够满足一般的黎曼度量。
	\item LCT通过优化与细化输入网格,能够同时满足目标各向异性和几何误差。当输入的各向异性存在剧烈变化,输入网格存在尖锐特征时,我们的算法都能很好地工作,可以得到高质量的网格和较高的计算效率。
\end{itemize}

在\textbf{第\ref{chap:polycube}章}中,我们构造多立方体结构的算法也是基于网格变形的。我们算法在变形过程中保证没有翻转和退化的网格单元。我们提出了一种新颖的法向光滑算子,通过调节算子的核大小可以平衡几何形变和多立方体结构的奇异性。我们将由法向光滑和法向对齐定义的能量合并来驱动网格变形,同时解决非单调(极限点)的问题。整个问题被自然地定义为一个无约束最小化的问题,并且通过一种非常简单高效的策略进行优化。我们的方法可以快速地获得高质量的多立方体映射。%在很多的复杂模型上,计算最多使用了4分钟(四面体的数量从20k到440k)。
我们与当下最先进的算法比较了算法鲁棒性,时间效率和可控制性,充分证明我们算法的优越性。
