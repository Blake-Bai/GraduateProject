\begin{cnabstract}
有限元方法是一项基本的数值技术,它在工程中有着广泛的应用。很多工程问题归结为解不规则区域上的偏微分方程,由于难以找到解析解,常常利用有限元方法在离散化的不规则区域上用数值解去逼近真实的解。近年来,我们已经有了高效的求解算法和计算能力强大的计算机,但仍然要花费大量的时间、甚至手动地把复杂的区域离散化为有效的有限元网格。在工业应用中,三角形网格和四边形网格都可以作为2维有限元分析的数据输入,但一般而言使用四边形网格进行计算具有更高的精度和更快的速度,然而,想要全自动地生成规则的四边形网格也要困难的多。

近年来,研究者们已经提出了一些2维四边形网格的生成方法,但对于复杂且不规则的2维区域,它们要么不够鲁棒或不能全自动化,要么会产生过多的奇异点。因此复杂的边界是对算法的鲁棒性、高效性和产生网格的质量的巨大挑战。为此,本文的思想是先对输入的区域做出简化,即变形为多正N边形结构(边界法向为正N边形之一的2维区域),并且得到两者之间的一个无翻转映射,再利用现有的算法在简化的区域上生成高质量的四边形网格,最后把网格映回初始的区域。另外,本文还采用了先进的优化算法在确定拓扑结构后提高四边形网格的质量。我们采用基于中轴线的几何分割方法生成四边形网格,验证了这一思想,能快速、鲁棒地产生奇异性可控的2维四边形网格。



\keywords{2维四边形网格生成\enskip 中轴线\enskip 多正N边形结构\enskip 无翻转映射\enskip 变形\enskip 网格质量提高\enskip}
\end{cnabstract}

\begin{enabstract}
The finite elements method  ( FEM ) is a powerful numerical technique which is widely applied in engineering.  Many technical problems are abstracted as solving partial differential equations on irregular domains. Because it is usually hard to find analytical solutions, FEM is employed to calculate numerical solutions in discretized irregular domains. Although there are efficient computational algorithms and powerful computers over the past years, discretizing complicated geometry to a valid finite element mesh is a time-consuming work, and may need complex user interaction.  Triangular and quadrilateral mesh are commonly used as data preparation in FEM, but in general the use of a quadrilateral mesh has better quality and converges faster than a triangle mesh. However, it is more difficult to generate a quadrilateral mesh automatically.

Recently,various 2D quadrilateral mesh generation method have appeared. However, they are either not fully automatic and robust enough, or will produce too many singularities for complex and irregular 2D domains.So,     complicated boundary is a great challenge to the robustness, efficiency and quality of quadrilateral mesh of all these methods. To solve this problem, our idea is simplifying the input first, i.e. deforming it to \todo{多正N边形结构}(each boundary normal of which belongs to one of the regular-N polygon), computing a inversion-free mapping between them at the same time, then using existed algorithms to generate a quadrilateral mesh of good quality, finally mapping the mesh back to the initial domain. In addition, an advanced mesh optimizing algorithm is employed to improve mesh quality after the topology is determined.  A medial-axis-based geometry decomposition method is used to prove our algorithm, which generates singularity controllable 2D quadrilateral meshes efficiently and robustly.

\enkeywords{2D quadrilateral mesh generation, medial axis, \todo{多正N边形结构}, deformation, inversion-free mapping, deformation, mesh quality improval}
\end{enabstract}
